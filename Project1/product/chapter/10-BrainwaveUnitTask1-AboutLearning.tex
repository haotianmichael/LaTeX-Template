\chapter{关于学习}
  
\section{探究活动}
    请四到五个成年人在每天早课开始的时候花不超过20分钟给班上的同学讲讲他们的学习方法。这样持续一周。务必确保每一个人都要讲一些不一样的东西。内容最好是一半比较严谨,一些比较有趣的。\par
    要让孩子们知道:学习无时不刻的发生在身边,涉及生活的方方面面。终身学习是很重要的。所以当那几个人谈到一些关于自己的工作、家庭甚至父母的时候就显得很正常了。你需要在过程中请教着几个大人知识、技巧和理解这三者的区别在哪里,最好能给出自己的例子。然后与孩子们展示中的关于这几个概念的理解做一下对比。看看是否有什么需要补充的,通常大人的经历会更丰富一些,因此提供的例子会更多一些。\par
    还可以提示大人们思考关于IPC个人目标。IPC个人目标是八项每一个渴望成为卓越学习强手的人需要掌握的品质。\par
    \begin{itemize}
      \item 探究(不满足于现状)
      \item 适应力(可以在不同的环境下使用不同的方法达到同一个目标)
      \item 恢复力(遭遇挫折不会放弃)
      \item 道德(有所为而有所不为)
      \item 交际能力(能够和别人分享合作 )  
      \item 主见(有自己关于学习的见解)  
      \item 合作(我们能够适应和团队作战) 
      \item 尊重(我们不只为自己考虑)  
    \end{itemize}
    在课堂上展示这八个目。’教师手册‘上也主要是对IPC个人目标的解释和讲解(该手可以在IPC会员区下载)。而且届时孩子们应该已经意识到了这些(前提是这八个目标在你们学校的教学体系中是被支持的)并且每一个孩子应该对这些有自己的见解(详细参见里程碑1脑波单元)。\par
    向大人们请教自身关于这几个品质的经验————这几个品质的重要性分别是在什么时候开始体现的,并且主要是因为什么?这些经验也可以记录下来,并作为资料展示在你们的展示区。\par
    最后也可以允许孩子们问一些问题,并整理其中一些比较具有代表性的作为知识收集的一部分。这些问题需要体现以下方面:\par
    \begin{itemize}
       \item  你最喜欢的学习方式是什么?你体验过的最成功的学习方式又是什么?
       \item  你觉得主要最困难的问题是哪种类型?最容易的类型呢?
       \item  你是如何改善和提高自己的学习效率的?
       \item  为什么说多加练习是实现学习目标的重要组成部分?
       \item  你最自豪的成就是什么?你认为现阶段最重要但是还没有实现的目标是什么?
       \item  学习对于你来说意味着什么?
    \end{itemize}

\section{记录活动}
     每当有一位课堂嘉宾离开的的时候,和学生们一起讨论他们从这位嘉宾身上学到了什么?鼓励学生寻找自身和嘉宾身上关于学习方法的相同点。老师应该思考:嘉宾们学习的不同方式与孩子在学习入口中观察到的不同类型的学习方式。这些嘉宾在学习过程中有没有使用到这些学习方法?老师也应该考虑一下IPC的的人目标和嘉宾们举的自身例子。这些例子和孩子们关于在课堂上举的例子有什么相似点和不同点?\par
     对这些嘉宾提出的学习类型作出检测,学生自主看这些学习的方式是否可以作为探究(找到答案)的例子或者作为记录活动(论证、展示、使用)的例子。然后将按照不同的分类进行分组。对第一个嘉宾老师可以做示范并作出模板,以后的每一个嘉宾都应该是学生自己分组寻找这些相似点和不同点。然后每一个组可以做一套可以用在探究活动和记录活动中的不同学习方法的列表,为了在以后的任务中参考。要经常不断讨论那个表并最终形成一个班级表,老师可以将这个表放在展示区。里面的一些观点和知识可以被作为未来单元学习和任务探究的对象。\par
     在每一位嘉宾离开之后,回过头来复习知识获取环节中的思维导图并适当做一些补充,主要是补充观点、问题甚至是这些就问题和嘉宾们谈论过后自认为合适的答案。\par



\section{个人目标}
     \begin{itemize}
       \item 交际
       \item 探究
       \item 尊重
       \item 体贴   
     \end{itemize}
