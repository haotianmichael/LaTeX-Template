\chapter{积极思考}
\section{探究活动}
    请同学想一次曾经没有准备好或者不想学习的经历。或许他们还能够感觉到难过、生气、累、紧张甚至害怕。和同学分享一次你自己的经历,在那段时间内你被消极情绪所影响,没有办法学习。\par
    请同学们自己画出一张图片,来表现出他们在那段时间的感觉。学生也可以在图片周围加上解释性的语句来表达自己的情感。然后将图片与班上的同伴交换,同伴可以在拿到的图片上写上一些积极的话。这些积极地话可以在粘滞便笺上剪切或书写,并附在图纸边缘。然后两个人就可以一起回顾和讨论彼此的消极情绪和得到的积极地鼓励的话。\par
    邀请志愿来分享自己的经历,然后全班一起讨论。旨在培养出一种好的品质:或许班上的每一位同学都是不一样的,也拥有自己的不一样的忧愁的事情和担忧。但是在面对困难的时候,所有都是平等的————那就是得到支持的机会和能力。\par
    现在老师分享一个属于自己的例子:在这段经历中,你通过积极调整自己的情绪让自己的学习变得很高效。问一下同学他们是否也有这样的经历。然后大家一起讨论一下这个话题,最后试着得到结论,主要是探究一下每一个人的行为和情绪是如何影响自己的学习效率的。\par
    向同学们介绍这个单词“心态”\footnote{原单词是mindset}————意思是我们处理事情的态度和方法。一个积极的心态会帮助我们的学习变得更高效。\par
    回想一下你曾经经历过的消极情绪和过程中别人给你的建议。讨论一下班级成员可以互相支持彼此的方式。

\section{记录活动}
    制作一副挂图,你可以在每一个讨论会开始的时候和同学使用这幅挂图。\par
    一列上写到:'我准备好学习了';另一列上写到:'我并没有准备好学习'。挂图的周围可以用孩子们的例子来装饰。\par
    一定要不能表现出偏向性(对于这两种心态),然后让孩子们自行选择将自己的名字填在那一列中。然后孩子们会给出这样做的原因。试着给那些没有准备好学习的孩子们提供一下帮助。如果有一两个孩子还是持续性的不能进入学习状态,你或许可以做下面的事情来帮助他们减轻忧虑:\par
    \begin{itemize}
      \item 慢呼吸
      \item 站立几分钟
      \item 想象自己状态好的时候的样子 
      \item 在一张纸上写上他们认为的忧虑、将这些纸放在一个盒子里以便待会儿再去讨论  
    \end{itemize}  
    作为这项任务的附加部分:同学们可以帮助你完成一个积极思考的集合然后将他们和学校的其他班级分享。好好思考一下你应该如何展示这些情绪,可以参考一下学习入口中的相关部分以及前一项任务来帮助你思考。或许你还可以考虑将这些与第7项任务中的学习联系起来。\footnote{指的是下一章的内容}\par
    典型例子如下:\par
    \begin{itemize}
      \item 一次角色扮演,大家可以探究一下消极情绪和如何克服消极情绪。积极的心态——一种很正能量的英雄角色。可以通过很多形式来表达积极的建议。
      \item  一首歌、一首说唱如何将消极情绪变成积极情绪。也可以说一下正能量的话。
      \item  通过大脑起床活动 (见下一章) 来帮助同学准备学习
      \item  熟能生巧——孩子们也可以展示自己的技巧(乐器展示、街头足球、讲故事等)然后讨论一下他们是如何练习的,在遇到困难的时候为什么没有选择放弃。
      \item  进入大脑,同学们可以讨论任务3中的大脑活动——学习过程产生的原理和神经元工作连接的机制。
    \end{itemize}


\section{个人目标}
    \begin{itemize}
      \item 沟通
      \item 合作
      \item 尊重
      \item 深思  
    \end{itemize}  
