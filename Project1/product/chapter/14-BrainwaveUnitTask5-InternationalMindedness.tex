\chapter{国际化思维}
    

\section{探究活动}
    提醒学生们上节课做过的‘连接’游戏。然后思考你们所在的班级和世界上的其他地方和国家都有什么关系。老师也可以先对班级成员认为和世界上的关系之数量作出预测。找出一张世界地图,然后标出你们所在的地区。用一根大头针标出来。然后大家一起在地图上画出你们认为和班级有关系的世界上的其他地方的位置。并邀请同学分享一下去过那个地方的经历,也可以分享关于这个地方其他的大家不知道的信息和知识。\par
    下面的信息可以在做连接的时候提供参考:\par
    \begin{itemize}
      \item 祖国(班上应该有留学生)
      \item 旅游过的地方
      \item 家人和朋友 
      \item 产品(食物、媒体)
      \item 知道的地方(通过以前的学习等)  
    \end{itemize}  
    回顾一下上节课的班级展示(上一个任务)。问一下同学为什么他们觉得国际化要被单独作为一个科目。请同学们自行探究一下国际化到底代表了什么和与之相关的学习类型。老师们可能希望查看Milepost 2中的国际主题的每个学习目标(可以在老师手册中IPC会员区找到)并且思考为什么这些对学习来说很重要。这些学习目标主要有一下:\par
    \begin{itemize}
      \item  了解祖国和所在国家的不同点和相同点
      \item  了解这些相同点和不通点影响人们的方式 
      \item  能够识别与自己不同但相等的活动和文化
\end{itemize}  
    回过头来看你们的地图和你们刚刚画出的连接。通过活动,邀请同学分享他们从中学到的东西并提示他们总结并阐述除为什么说世界上其他地区的文化和知识点对我们的学习过程也是很重要的。\par
    如果可能的话,和其他国家的学校建立联系。(你可以在IPC的会员区找到其他上IPC课程的外国学校)这种方式会让你能够分享你的学生所学的成果并获得比较有质量的反馈。\par
    社交媒体让分享工作变的非常容易。除了电子邮件之外,你还可以使用视频展示,通过你们学校的网站在在线展示工具比如\href{http://www.prezi.com/}{Prezi}.\par
    告诉同学们将会有一份他们自己的学习展示文件资料发送到另一个国家的学校进行展示。在地图上找到这个国家,并和同学们一起讨论关于这个国家都知道的知识和细节。他们会认为这个国家的学生和自己有任何相似之处吗?或者有什么相同的地方?甚至可以讨论是狗有相同的学校作息表?\par
    同学们可以讨论和计划的信息类型,以便于让外校的同学更好的认识自己。思考他们使用的分享信息的方式。比如说视频、图片、简历等。
\par

\section{记录活动}
    同学们可以自己或者组团完成这项自我介绍资料。资料可以在整理好之后在任务结束最后发送给伙伴学校。伙伴学校也做同样的事情,这样你们就可以互相分享自己学生的学习和自我介绍了。\par
    仔细看一下伙伴学校发送过来的文件,讨论在孩子们的日常生活中有没有相似点。鼓励孩子们思考他们从中了解到的东西,这种国际化的学习方式好处在哪里?一起准备造句:"国际化思维很棒,这是因为……"\par
    同学们可以自己继续探究和讨论分享伙伴学校的同学资料从中找到一些更加有意思的联系。\par
    国际化的学习方式在整个IPC教学中都是很重要的一块。在很多学习任务中同学们会被要求用不同的视角来探究所学的内容。不同的视角值得就是从祖国(出身的国家)和所在的国家或者以前生活过的国家。国际化也是一个很独立的学科,在每一个IPC主题结束的时候。国际化任务帮助发展孩子们对多种文化和国家的理解,并鼓励他们多参加国际化的活动。\par
    你们或许想参加一个国际化的活动。这样也可以长时间的专注于这些知识点上。下面的网站有详细的介绍:\par
    \begin{itemize}
      \item \herf{http://www.gogivers.org/}{gogivers}是一项针对小学的英国计划,旨在帮助他们与当地社区建立积极的联系。
      \item \herf{http://www.freethechildren.com/}{freethechildren}是一个致力于解决贫困和实施可持续发展计划的国际慈善机构(如“采用村庄”)。 该网站提供照片画廊和教师资源。
      \item \herf{http://www.unicef.org/}{unicef}联合国儿童基金会是一个国际慈善机构,为每个儿童的权利而斗争。 他们与政府和其他组织合作,帮助解决贫困,暴力,疾病和歧视问题。
      \item \herf{http://www.actionaid.org/what-we-do/education}{actionaid}是一家国际慈善机构,在全球45个国家开展工作,以解决贫困和不公正问题。 他们的许多项目包括与社区合作以改善儿童接受教育的机会。
      \item \herf{http://www.wateraid.org/}{waterald}是一家国际慈善机构,旨在通过改善世界上最贫困社区的用水,卫生和环境卫生来改变生活。
      \item \herf{http://www.oxfam.org.uk/education/global-citizenship}{oxfarm}开展了一系列全球公民项目,以鼓励儿童参与并参与重要的全球性问题。
    \end{itemize}  
    
    \begin{note}
      一份给家长的资料:"帮助你们的孩子理解他们所在的世界"可以在IPC会员区下载。里面包括了家长可以与孩子一起做的建议和活动,以促进国际社会的发展。也是你的学生们回家后可以和家长一起做的很好的家庭作业。
    \end{note}  

\section{个人目标}
    \begin{itemize}
      \item 适应力
      \item 交际 
      \item 尊重
      \item 体贴  
     \end{itemize} 
