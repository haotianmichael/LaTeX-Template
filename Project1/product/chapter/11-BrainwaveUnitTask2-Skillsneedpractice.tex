\chapter{熟能生巧}

\section{探究活动}
    让学生思考自己教别人一样技巧的时候可能会用到的不同的教授方式。把他们的发言用表格的形式记录下来。列表中应该包含:\par
    \begin{itemize}
      \item 总结一套教学指令
      \item 一场简短的报告
      \item 做简单的示范 
      \item 制作一份教学视频  
    \end{itemize}  
    
    小组为单位或者两个人为单位,学生将会以孩子的身份向自己熟悉的大人教授一项技能。成人对象通常包括助教、老师、学校的其他员工、父母等。关于这项技能,应该是在课堂上孩子们比较熟悉但是成人们一般不会经常做的技能。并确保即将教授的这项技能掌握起来还是有一些难度。\par
\subsection{一些活动的例子以及随附的在线资源}    
    
   \begin{itemize}
      \item  将彩色塑料管编织成钥匙圈和友谊手镯--\href{http://www.scoubiguide.co.uk/}{详见ScoubiGuide}
      \item  Rainbow Loom是一个受欢迎的游乐场热潮,孩子们可以将橡皮筋编织成手镯,项链和饰物。--\href{http://www.rainbowloom.com/}{详见rainbowloom}
      \item  Stuff Works提供有关一些流行的悠悠球技巧的说明--\href{http://entertainment.howstuffworks.com/easy-yo-yo-tricks.htm}{详见StuffWork}
      \item  口袋妖怪是一个交易卡片游戏的例子,玩家使用他们收集的宠物卡片互相争斗。 通过理解战术以及可以使用的权力和能力的不同组合来获得技能。--\href{http://www.pokemon.com/uk/pokemon-tcg/play-online/tutorial/}{详见Pokemom}
    \end{itemize}  

   \begin{note}
     老师要经常留意那些学校允许学生玩的娱乐和游戏活动,有时候这些东西会很有用。
   \end{note}  
   

   回顾一下在这节任务开始的时候列出来的学生的教授方式列表。让每一个小组随机从列表中选择一项方法(或者由老师随机选择)。\par
   小组开始探究他们选择的教授方法,并计划准备如何使用这种方法将技能交给大人。根据他们选择的方法,孩子们可以通过照相、采访和视频整个过程的方式来收集材料。\par
    对学生来说,要能够在教授过程中充分理解‘熟能生巧’的道理。反映在活动中,就是被教授的成人们都需要一个‘可以带回家;练习’的技能包裹。这样他们回家之后也会自己进行练习。并且要准备一个到两个跟踪环节,这样在掌握技能的同时大人们也可进行学习评估甚至自己记录活动的过程。\par
    

\section{记录活动}
   当成人的学习有进步的时候,老师应该提示学生如何评价这些成就。孩子们或许已经熟悉了通过使用IPC学习评价计划中孩童评价标准来进行自我评价和对同学进行同级评价。该计划提供了三个级别(初级、中级、精通),孩子们使用这三个级别对自己的成就作出评价,并根据该标准给自己定下下一阶段的目标。学校官方的使用方式,可以再IPC官网上的会员区找到下载的文档。\par
   通过这项活动,学生最后能够使用自己定义的三个标准对别人的学习进行评价。这个可以通过和别人合作来共同总结使用,也可以参考以前的评价技巧。然后使用总结好的评价标准对大人们的学习进行评价。\par
   学生小组内合作记录他们调查的结果和最后得出的结论。总结反思一下迄今为止探究过的所有不同的记录学习的方法,这时候学生应该能够将这些信息以自认为最高效的方式展示出来。回归课堂,复习一下最后的结论,讨论这次活动学到了什么。\par
   需要讨论的主要问题包括:\par
   \begin{itemize}
     \item 在一项技能的掌握过程中,勤加练习有多重要。
     \item 一般来说那种教学的方式会更加有效。
     \item 大人们心中好的教学方式和孩子们心中的想法是一样的么。
     \item 学习效果是否和年龄有关。
     \item 有什么关于使学习变得更容易或更难的不同方法吗?
     \item 如果学生将会有另一次教授大人技巧的机会。学生们将会怎么准备这次教学活动?对从初级到中级再到高级,学生们对大人们的学习建议会是什么?这次活动中学生的探究法方式会有什么不同?
     \item 在学生看来,勤加练习对掌握一项技能的重要性占多少?尽管对有些人来说已经掌握了这项技能。
   \end{itemize}  

   \par
   在课堂上指出:演员、音乐家、运动员等所有在特定领域的人们都会花大量的时间去熟练自己的专业技能。学习过程从来都不是自然形成的。Malcolm在自己的著作《局外人》中写道:每个人都需要花费10000小时来掌握自己的专业技能。\par
   老师可以举办一项活动,让孩子们在表格中记录每一个他们知道的在自己领域练习了10000小时然后成功掌握了专业技能的成功人士。\par
   
   \begin{note}
     当谈及学习的时候,要不断将“但是”这个词的作用记在心里。如果有学生说自己不能掌握这项技能的时候,要不断尝试并且确信通过不断的练习和付出,最后一定会掌握这项技能。老师们可能会想要和同学们一起分享那个最受欢迎的故事:《不能跳舞的长颈鹿》。这个故事的主题就是关于坚持和耐心。故事讲了一个笨拙的长颈鹿通过努力最终在舞林大会上技艳群雄,而在此次之前他被告知根本不适合跳舞。很简单的小故事但是很正能量,很适合在课堂上讲。
   \end{note}  

\section{个人目标}
   \begin{itemize}
      \item 沟通
      \item 合作
      \item 探究
      \item 深思
   \end{itemize}  
