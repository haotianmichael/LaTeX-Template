\chapter{知识图谱}
    IPC为里程碑1,2,3写了三个脑波单元(学习的艺术)。每一个单元帮助老师和学生了解学习是什么,学习意味着什么,学习有什么内容可谈论,如何评价学习成果,在课堂上不同类型的学习会有什么样的实践结果,如果将学习地点变成户外会怎样呢等……
\par
    我们建议在每个学年开始时教授这些单元的各个方面,让孩子们学会学习的思维模式。\par
\section{IPC 自我检查}
    IPC自我检查文档和进程资料对所有的会员学校都是开放的,可以在我们的会员区进行下载。它通过设立与学习者,教师,领导者和社区相关的初级,发展和掌握水平来指导学校完成IPC实施和发展的9个关键领域,即“底线九”。\par
    在接下来的内容里自我检查会跟着涉及到很多的细节(比如:国际意识、个人目标、班级活动、评价、不同类型的学习、关于学习的最新的研究等)并且就如何和不同的利益相关者共同发展这些领域给出了详细的指导。

\section{关于学习}
    ‘我们学校是IPC学校’和‘我们使用IPC来帮助提高孩子们的学习水平’这两者之间有什么去吗?抑或是我们有一些钻牛角了。我们认为这两种称述之间的不同本质上在于IPC到底是什么。 \par
     经验告诉我们那些最成功的学校永远都将提高学习质量深深地嵌套在自己的教学体系中————所以这些学校中的孩子在学术、私人和国际化领域学习效果都是最好的。\par
     但是什么才是学习、我们如何才能真正掌握孩子们是否、在什么时候、用哪种方法才能从学习中获取知识?本单元的学习同时帮助老师和学生去深入思考这个问题,然后总结出一套学习语言供所有利益相关者使用。  \par
     这套学习语言要超过语义上的语言本身。家庭作业和居家式学习之间的区别是什么?跟孩子们谈论他们的学习任务和跟孩子们谈论他们的学业之间的区别是什么?一个孩子们的家庭作品展和旨在提高学习能力的展示两者之间的区别是什么?老师们可以在自己的学校尝试着实验这两类问题之间的不同————一定会得到不同的结论并效果显著。\par
     越来越多的数据证明:了解到我们的大脑是如何运作的并将这些成熟的理论运用到学校的个体教学中来可以使学校的教学效果有显著的提高。总结一下就是学习本身也是可以学习的。本单元的学习不仅帮助老师和学生深入了解大脑的运作机理,还提供了一个提高每一个人元认知的机会————这些是思考、反馈和提高学习能力的必要条件。 \par
 
\section{IPC课程的学习过程}
     整个IPC课程围绕一种理论过程展开,这种过程持续关注5到12岁的孩子学习的方式。学习到整个课程内容对孩子们来说是远远不够的,他们还应该认识到IPC这种教学方式的好处和重要性。 \par
     

\subsection{学习入口}
    这是整个IPC课程的开始环节,过程也很有意思和刺激。 \par
    学习入口的目的是为了让孩子们思考和商讨即将要学习的内容。

\subsection{获取知识}
    本环节帮助老师们了解孩子们就即将要学习的主题已经掌握了些什么。然后在孩子们想要学习的东西基础上为他们个性化定制课程,从而对原来的课程进行适当的删减。这样做加强了学生旧知识点和即将学习的知识点之间的联系,从而然然孩子们取得学习的主动权。\par

\subsection{主题阐述}
    本环节在整个课程开始之前为老师和学生提供了单元的学习概要,这样做的好处是可以促进学习主题和概念之间的联系。\par

\subsection{探究活动}
    每一个主题学习都会有探究活动,该环节被设计的目的是来确保孩子们能够以正确的方式来获得信息,可以尝试多种学习方法比如角色游戏、数字学习、图书馆查阅等。这些方法中有一些方法之间是有联系的,还有一些方法被设计出来发展学习个体的探究能力和恢复能力,进而促进和改善IPC的个人目标。

\subsection{记录活动}
    该环节帮助孩子们加工和展示他们在探究活动中获得的各种信息和知识。获取这些知识的方式旨在锻炼他们的各种能力和兴趣,从而促进其记录学习的能力。记录学习的方式可以有很多种,比如说数字设备记录,戏剧形式、作曲、地图、图片、实验、艺术作品等等。

\subsection{学习出口}
    本环节结束整个章节的学习。最后让孩子们在他们以前的展示中画上新学过的知识点。这种方法有助于提醒孩子们联系起已经完成的主题之间的知识点。并提供了时间和机会来对已经学习的内容作出自己的理解,然后以小组为单位进行思考。学习出口提供了和父母接触的绝好机会,也可以邀请他们一起庆祝本单元的顺利结束。 \par

    IPC自我检查标准7:\textbf{IPC的学习过程}为IPC的教学结构提供了更详细的指导。


\section{IPC的学习领域:学科、个人和国际学习}
   IPC学习单元结合了三个领域:学科、个人和国际。

\subsection{学科学习}
    IPC课程是严格按照选择好的主题来组织的。主题(比如主题‘巧克力’)反映了一定年龄阶段孩子的兴趣和关注点。这样的好处在于可以直接促进和驱使他们主动式的学习并发现更多有关的内容。在这些主题中,孩子们可以通过不同的学科视野来学习不同的内容,这些学科都包括:科学、历史、地理、音乐、艺术、信息和通信技术\footnote{ICT是information and communications technology.的简写}、科技、体育、社会学以及国际学科等。实际上在IPC,孩子们不需要深入的学习地理这门学科。我们聚焦于不同主题在地理方面的知识点比如‘奥运会’、‘假期’、‘健康’。用这种方式组织学习也可以让孩子们初步了解学科之间的相互交叉和相互独立。从而对即将学习的知识点有一个大的图谱,在图谱中进行不同知识点之间的连接,不同知识点之间的交叉并且从不同的角度来商讨所学习的内容。在线规划部分(见会员区)帮助学校计划需要执行的教学单元并并跟踪单元内每个科目学习目标的覆盖范围,这样可以确保课程的广度和平衡性。\par
    IPC自我检查标准8:\textbf{通过相互独立但是又相互依存的主题进行主题实施}对主题学习提供了更多的指导。

\subsection{个人学习}  
    IPC的个人学习目标指的是那些我们认为21世纪对孩子来说最重要的学习品质和学习倾向。\par
    IPC总共有8种个人学习目标:探究,适应力,道德,沟通,体贴,合作,尊重和适应性。体验和练习这些特定的学习目标已经被固定在每一单元的学习任务中并在每一个单元学习结束的时候提醒老师们进一步优化主题。课堂教学和学校的各个方面都应该体现出对这些品质不断追求的努力。将学校的课堂教学和这些个人目标联系起来有助于学校建立教学目标上的头统一性并帮助他们更容易向别人解释学校的整个教学计划。   \par
    IPC 自我检查标准2:\textbf{关于我们正在帮助发展的儿童种类的共同愿景}为如何进一步发展和适应IPC个人目标提供了更详细的指导。    

\subsection{国际学习}
    在帮助孩子们拥有逐渐成熟的国家,国际,全球和跨文化视角、发展一种国家化意识方面,IPC具有独到的优势来定义国际化学习的目标。每一个IPC单元都有涉及到国际化,在不同的学科之间以学习为主的活动除了鼓励做更多的全球视野的活动之外,还帮助孩子们发展一种国际意识并对自己、对自己生活的社区、世界有了更深的认识。除此之外,每一个IPC主题不仅鼓励学校从两个不同的视角:居住国家(学校所在的国家)和祖国(孩子们的祖国)来探究单元内容,还将包含国家化内容的主题作为一个学科。有时候居住国家和祖国是一个国家,这个时候学校就应该找到至少一个国家来探究该单元的学习内容。\par
    IPC自我检查标准4:\textbf{国际意识}为学校的国际化教育提供了更多详细的指导。

\section{IPC的学习类型:知识、技巧和理解}
    学习目标是IPC创建之初的信念基石。学习目标定义了学生应该学什么,能够做什么,在学术个人和国际领域如何发展。\par
    我们相信知识、技巧和理解力之间的不同对孩子学习的发展起着至关重要的作用。我们还相信这三者都各自有属于自己的独特的特征,这些特征对每一个个体的规划、学习方法、教授技巧、评价和报告都有着某种影响。这些特征之间的差异影响是深远的,值得深入考虑。   \par
    

\subsection{知识}
    \textbf{知识}指的是真实的信息。尽管要回忆起这些知识并不是简单事,但相对来说知识本身是比较容易能够传授和评估的(通过智力测试、课堂测试、单项选择等)。你可以要求孩子们去研读需要学习的知识,但是你也应该告诉他们那些他们需要知道的知识。知识本身是不断更新迭代的,这对于学校来说是一个挑战,因为学校必须能够选择在特定的一段时间内适合孩子们了解和学习的知识。 \par

\subsection{技巧}
     \textbf{技能}指的是那些孩子们能够做的事情。技能需要在实践中学习并需要时间去掌握。技能的特点是你训练的时间越长,那这项技能你掌握的越熟练。技能也是可以传授的,往往比知识更稳定————这几乎适用于所有学校的教学科目。IPC教学评价是建立在孩子们实战能力的(技巧)的基础之上的。\par
 
\subsection{理解}
     \textbf{理解}指的是对一些概念思想的逐渐把握和发展,换句话讲就是我们经常说的“灵光一现”的时刻。人的理解是不断发展的。没有人在这方面是真正的权威,但是IPC单元学习可以提供的,是一个全新的体验,这种学习体验能够帮助你加深你的理解力。\par
     我们认为知识、教育和理解是作为一个整体来运作的,而没有优先顺序————每一种不同的学习类型包括又超越了另外几种。但是每一类型的学习又有自己独特的学习差异,这些差异能让孩子们在学习的过程中体会到并走出区分到底应该是哪一种类型的学习,并意识到不同类型的学习对课堂教学的影响是什么。\par
     IPC自我检查标准5:\textbf{知识,技能和理解的意义和发展}为这一章节的学习提供了更多的细节额指导。 \par

\section{学习评估}
    IPC课程被设计来帮助老师们、学生们更加高效的学习、更加有趣的学习。但是表面上确保学生们在学习是远远不够的,我们需要一些标准来确保学生们是确实学到了东西。因此,IPC设立了为每一单元的学习设立了评价制度。\par
    这项制度通过在IPC学习目标中总结出来的关键技巧来提供:老师的评估、学生的自我评估。一共分为三个部分:   \par

\subsection{教师评估准则}    
    这一项是很重要的指标,它帮助老师们确定自己的学生正处在学习的哪一个阶段:\textbf{初步}、\tetbf{发展}、\textbf{精通}。

\subsection{学生评估准则}
    上述教师评估准则的孩童版本。在合适的年纪,供孩子们自我评估或者对同年级的孩子们进行评估。

\subsection{学习建议}
    详细的学习活动和建议,可以在课上进行分享也可以和父母进行分享。这样做也可以帮助孩子们的学习从一个阶段到进一步前进。这构成了反馈循环的最后一部分 - 前进到下一步并改进学习。\par
    
    为了支持整个项目的开展,我们提供了一个在线追查工具,在这个工具上你可以记录不同学科中孩子们学习和技巧的进度,并评测在那一个阶段。你也可以记录往常的评论、观察到的细节、和一些存档进度,然后将这些资料整理好之后和孩子们的父母一起分享。使用IPC评估工具需要专门的技巧,所以教师们和学生们得不断练习。 这些单元将帮助教师以适当的方式将课程引入课堂。\par

\section{关于学习的研究}
   最近神经科学和认知科学研究论文的激增这件事引起了学者和领导的关注。在这些论文中一定会有一些对那些想要提高学习能力的人是很有帮助的。它提供了在课堂上使用的新想法,并进一步解释为什么我们的一些直观但以前难以证明的方法与这些理论一样有效。   \par
   但也有可能过早地加入这个潮流。许多作家和教育家已经开始宣传那些并没有足够的理论依据的教学方法和技巧————你或许也听说过,就是媒体上说的‘神经科学的偏见’。  \par
   尽管我们已经学到了很多知识,但是还有太多我们不了解的领域——即使是脑科学的科学家们也非常反感被要求就教学给出自己的建议。也正是因为这个原因,我们鼓励学校和老师们将攻读关于学习的学术论文作为教学体系中不可分割的一部分。通过攻读、研究这些论文,老师们便可以搞懂学习过程本身的本质,然后将所学到的理论应用到课堂中甚至是自己的孩子身上。\par
   IPC自我检查标准7:\textbf{IPC的学习过程的实施}和本部分的学习和研究有关。\par

   我们并不能覆盖所有有关学习的学术论文,但是这里有一些你可能有兴趣想要了解的主题和趋势:
\section{成长型思想状态}
    '固定'和‘生长’理论的概念最早是来自于斯坦福大学心理学家Carol Dweck的一项研究。她最著名的一项研究将所有的孩子分成为两部分并对他们进行测试,在这项测试中故意让大部分孩子赢得测试。对其中任意一组的孩子表示肯定并表示他们成功的原因是因为他们很努力,对另外一组孩子也表示肯定但告诉他们是因为他们很聪明。然告诉这两组孩子即将会有一场新的测试,但是不一样是这一次测试孩子们可以自己选择是让测试题更容易一些还是更加难一些。最后那一组号称是努力的孩子几乎都选择了更难的测试,而那一组‘天赋异禀’的孩子80\%都选择了简单的测试。\par
    这位心理学家的研究结果强调了被她称之为‘固定’思维模式‘和’成长‘思维模式的不同之处,前者更容易程式化自己的行为,更容易放弃所以总是不能完成自己的任务;而后者更加有目的的学习,并且相信智力和能力都是可以发展和培养的,所以更加喜欢挑战。\par
    这项差异传递了一种信息:应该崇尚过程而不是能力。\par
    该心理学家在一篇文章中提到大脑就像是肌肉,需要不断的锻炼来边的更强。这一切都有科学理论的支持。在这篇文章中她解释了神经元是如何连接的,也正是因为这种连接和不同神经元之间的通信人们才能不断的学习和解决问题。神经元之间很难连接,当我们开始学习新的东西和不断思考的时候,这些连接会创建和不断加强。\par
    在你学习的过程中,大脑不断寻找过去的知识点和新学习的知识点并在两者之间创建连接,这也是\textbf{获取知识}这一环节很重要的原因。它加强了新旧知识之间的连接。   \par

\section{元认知}
    元认知指的是关于学习的本质。或者说成是学习的构成因素。\par
    在这里我们参考一位教育界的研究院John Hattie的研究成果,他发明了一种方法可以通过对元的分析直接对学习的作用进行排名。他提出了超过900中元分析,其中包括超过50 000种对24亿在校学生的研究成果和图表。他在书籍《有远见的学习》(2009)中确定了150个课堂干预措施并按有效性列出。元认知排在列表中的13名。Hattie在他的书中解释道:“当一个学生面临着论难的任务的时候,元认知发挥的作用要比智力发挥的作用更能直接对学习的结果造成影响”。\par
    思考如何学习是很重要的。在《让学习更加有效》一书中哈佛大学的研究员开发了一套思维程序,不仅可以帮助学生学习还能够教会每一个人如何掌握学习的技巧。作者指出对老师来说正确并清晰的表达出我们想要的思考模式是一回事,但是对学生来说培养一种对思考模式价值的重视意识是另一回事。\par
    在学生们进行学习的反思的时候,'慢思考'(Gay Claxton在自己的著作中使用的专业术语)也是老师需要考虑在内的。当遇到困难和一些复杂的东西的时候,大脑会接收到大量的信息,然后出于停滞状态————这种停滞状态是用来处理接受到的信息。而这个信息消化和沉淀和过程被称为是一次‘满思考’。这也是为什么在单元学习过程中将知识获取环节和主题阐述环节包含在内的原因所在,因为这样孩子们就可以不断的‘满思考’————反思自己的观点和问题、并一起想出问题的答案。\par
    在每一个单元的学习过程中,你都能发现有很多帮助老师和学生来思考学习、认识学习的机会。

\section{记忆}
    Daniel T. Willingham,一位维吉尼亚大学的心理学家在自己的书籍《为什么学生不喜欢学校》中探究了记忆的过程。(2009)\par
    他认为大脑是不擅长于思考的,如果有机会大脑便会依靠记忆拜托主动的思考。从广义上讲,人类有两种记忆模式——工作记忆(短期记忆)和长期记忆。我们的工作记忆在大脑中的空间是很少的————一般来说大大脑一次只能存放3-7块记忆,而且这些记忆的更新速度也很快。有效的学习过程指的大脑能够将短期记忆的内容传送到长期记忆的存放区存放,并经常可以检索和恢复这些记忆(也就是可以记住)。这可以通过使用诸如“分块”之类的记忆辅助工具,以及通过引入新学习和巩固现有学习的平衡来实现。 \par
    Daisy Christodoulou 在她的书籍《关于教育的七个谜团》中补充了Willingham的论述。在她的解释中,长期记忆是能够存储大量的记忆的,而这些记忆按照‘块’存储在一起。当我们掌握新的这个领域的知识和新的内容的时候大脑将这些知识吸收并转化到这个‘块’中。这也意味着如果我们掌握了这个领域的很多知识,那么掌握的越多,吸收和转化该领域知识的速度就会越快。Williamham也说过:“那些越博学的人只会变的越博学”。\par
    不仅仅是知识值得我们研究,技巧本身也需要深入研究。Willham认为尽管孩子们似乎已经掌握了一些并不需要在提高的技巧,但是不断的持续的练习还是很有必要的。他告诉我们:“大脑的运转已经是不自觉的”而正是这种不自觉加强了那些需要不断练习才能掌握的高超技巧,也正是这种不断的练习保护我们的记忆不会快速遗忘并不断转化变成长期的永久性记忆。\par
    
\section{积极情绪和消极情绪}
    消极的情绪会抑制学习过程的产生。积极情绪会促进学习过程的产生。所有强烈的情绪都会留下记忆的痕迹。这也是IPC想通过我们单元的教学看到的“积极的学习、有效的教学、伟大的趣事\footnote{great fun--专有名词,后文会提到}”。\par
    课堂实践在这里又重要的意义。\par
    首先,在学习过程中的深度参与有助于产生积极的经历。Mihaly,一位大学的心理学教授,也是‘流动’概念的创建者,‘流动’是伟大的趣事的组成部分。提到这里的伟大的趣事指的是对一项活动严格的参与感和因此而取得的成就感。需要注意的是,这些为了学习而准备的活动,需要非常严谨的理论和方法来支持进行。\par
    其次,Willingham指出记忆是我们思考的结果。而我们需要学生思考的东西就是学习。所以当我们组织一堂有意思的、生动的课程的时候,确保学习过程一定清晰、可以相互分享、可以不断回顾并反思这很重要。老师们要不断问自己孩子们正在学什么而不是问自己孩子们正在做什么。因为学习本身是很重要的。\par
    最后,我们中的大多数人都在学习过程中经历过迷茫期,并且将持续经历了压力对我们学习能力的严重影响。Daniel教授在他的书《情绪智力》中清楚的展示了压力是如何阻碍学习过程的,压力会在大脑中产生一种叫做‘情绪化阻塞流’的物质。而教授提到学习的最好状态应该是‘放松之余的警惕’。处在压力中的孩子是不能够学习的。这事简单不过的道理。也正是因为如此,对老师们来说创建一个轻松的课堂学习环境对孩子们的学习效果是很重要的。\par
    IPC自我检查标准6:\textbf{孩子们的刻苦和老师们的期望}   IPC自我检查标准1\textbf{聚焦于提高学习能力}提供了这方面的细节。


\section{学习评估}
    “不管在什么情况下,提高学习效果,让孩子们成为终身的学习者的最有力的教育工具都会是格式化的评估形式。”Shirley提到。\par
     Shirley的论述依靠了Hattie的研究支持。Hattie的研究对格式化评估的关键要素做了排名,主要包括:
     \begin{itemize}
       \item 评估让学生保持知情权(学生们了解自己在学习什么,有一个明确的目标、当然也可以自己检测) 
       \item  提供了格式化的评估
       \item  反馈
     \end{itemize}  
     没有捷径可走————如果我们真心想要提高学生们的学习水平和质量,那么我们必须对评估保持同样的热情。

\section{难做的理解}
    因为在概念上的难以把握,所以’理解‘一直以来很少被运用到课堂中来。但是从教育角度来讲,理解已经被证明有很大的作用并完全可以作为一些教育主题的最终目标。也正是以如此,很多课程的创建者和作者已经开始聚焦于理解力的实践和评估,尽管他们还不清楚理解的本质到底是什么。\par
    教育工作者们已经理解了”理解“好多年了。在1933年,John Dewey在自己的书中将理解描述为”学习者寻求知识的结果;一件事情、一项任务、一个场景在和其他事情的关系中体现出的东西;一样事情如何运作或者其功能是什么;伴随者的结果;导致的原因;用途之间的联系等;结果是理解的核心和本质“。\par
    Willingham对理解的认识是:理解实际上是一种伪装的记忆,当学生们开始通过自己的记忆(明白本质的东西)理解一样东西的时候(不知道本质的东西)。他解释道说理解一样新的概念更实际的说法应该是将新的概念吸收和转化为长期记忆的一部分,可以通过和原来的记忆进行能够比较或者换一种方式去思考。这些都可以看成是理解的一种。\par
    Willham还提到了必须对孩子们的学习情况抱有的实际的态度,他们学习的天花板是什么,最有效的学习速度有多快。否则教学和评估只会是虚假的把戏。\par

\section{大脑和身体的联系}
    一直以来关于大脑和身体的研究不绝于耳,我们经常能够听到说好的饮食、健康的身体、积极地 锻炼以及充足的睡眠会让我们的大脑运作更加快速。\par
    总的来讲,平衡饮食对大脑的发育还是很重要的,饮食应该包括各种水果和蔬菜。你的大脑需要持续不断的能量补充,而主要的能量来源就是碳水化合物中的葡萄糖。\par
    锻炼也被认为是大脑发育的重要组成部分。锻炼不仅仅确保了大脑拥有充足的氧气。研究表明锻炼还能够有效的促进记忆功能。David Bucci以为脑科学的部门助教解释说:促进记忆的主要原因是在锻炼的时候大脑能够和身体各个机能共同促进,从而产生一种能够加强大脑中连接的物质,这种物质加强了记忆之间的关联,促进了大脑的发展。研究报名年轻的时候锻炼是最好的。(锻炼你的大脑  http://www.medicalnewstoday.com/)\par
    睡眠也是很重要的,睡眠的时候大脑会自动进行补充和疲劳缓解并进一步转化白天发生的学习过程中掌握的知识点。除此之外,一项美国研究小组的研究显示睡眠的重要性,因为正是在这段时间内,脑细胞缩小并打开神经元之间的间隙,让液体清洗干净大脑。(睡眠清洗你的大脑  http://www.bbc.co.uk/news/health-24567412)\par

\section{社区}      
   学校应该经常和学生的家长,监护人保持良好的的联系和合作来让他们更好的参与自己小孩的学习过中。当学校使用学校的语言与父母分享时,孩子们可以进一步学习,这样父母就知道如何与孩子谈论他们的学习,赞扬他们对结果的努力,并鼓励他们每天反思这种学习。\par
   IPC为父母们制作了一些讲义来配合学校更好的合作家长。下面的文档可以在文档去自行下载:
   \begin{itemize}
     \item 帮助孩子们理解他们在当今世界上的地位
     \item 帮助孩子们学习————知识、技巧、理解
     \item 帮助孩子们学习———应该和你孩子讲的十件事情
\end{itemize}  

\section{书单}
Seven Myths About Education, Daisy Christodoulou, Routledge,2014 \par
 Outstanding Formative Assessment, Shirley Clarke, Hodder Education, 2014 \par
Hare Brain, Tortoise Mind, Guy Claxton, Fourth Estate, 1998\par
Flow: The Psychology of Happiness: The Classic Work on How to Achieve Happiness, Mihaly Csikszentmihalyi, Rider, 2002\par
How We Think, John Dewey, Martino Fine Books, 2011\par
Mindset: How You Can Fulfil Your Potential, Carol Dweck, Robinson, 2012\par
Emotional Intelligence – Why It Matters More Than IQ, Daniel J. Goleman, Bloomsbury, 1996\par
Visible Learning: A Synthesis of Over 800 Meta-Analyses Relating to Achievement, John Hattie, Routledge, 2008\par
Making Thinking Visible: How to Promote Engagement, Understanding, and Independence for All Learners,Ritchart, Church and Morrison, Jossey Bass, 2011\par
Why Don’t Students Like School?, Daniel T. Willingham, Jossey Bass, 2009\par
Articles You Can Grow Your Intelligence, Carol Dweck, 2009. (Available to download from mindsetworks.com/free-resources)\par
Sleep ‘cleans’ the brain of toxins, James Gallagher, 2013. (Available to download from bbc.co.uk/news/health-24567412)\par
Informing Pedagogy Through Brain-Targeted Teaching, Mariale Hardiman, 2012. (Available to download from jmbe.asm.org/index.php/jmbe/article/view/354/html)\par
Exercise Affects the Brain, Petra Rattue, 2012. (Available to download from medicalnewstoday.com/articles/245751.php)\par








