\chapter{知识图谱}
    IPC为里程碑1,2,3写了三个脑波单元(学习的艺术)。每一个单元帮助老师和学生了解学习是什么,学习意味着什么,学习有什么内容可谈论,如何评价学习成果,在课堂上不同类型的学习会有什么样的实践结果,如果将学习地点变成户外会怎样呢等……
\par
    我们建议在每个学年开始时教授这些单元的各个方面,让孩子们学会学习的思维模式。\par
\subsection{IPC 自我检查}
    IPC自我检查文档和进程资料对所有的会员学校都是开放的,可以在我们的会员区进行下载。它通过设立与学习者,教师,领导者和社区相关的初级,发展和掌握水平来指导学校完成IPC实施和发展的9个关键领域,即“底线九”。\par
    在具体的领域里检查文档涉及到很多的细节(比如:国际意识、个人目标、班级活动、评价、不同类型的学习、关于学习的最新的研究等)并且就如何和不同的利益相关者共同发展这些领域给出了详细的指导。

\subsection{关于学习}
    ‘我们学校是IPC学校’和‘我们使用IPC来帮助提高孩子们的学习水平’这两者之间有什么去吗?抑或是我们有一些钻牛角了。我们认为这两种称述之间的不同本质上在于IPC到底是什么。 \par
     经验告诉我们那些最成功的学校永远都将提高学习质量深深地嵌套在自己的教学体系中————所以这些学校中的孩子在学术、私人和国际化领域学习效果都是最好的。\par
     但是什么才是学习、我们如何才能真正掌握孩子们是否、在什么时候、用哪种方法才能从学习中获取知识?本单元的学习同时帮助老师和学生去深入思考这个问题,然后总结出一套学习语言供所有利益相关者使用。  \par
     这套学习语言要超过语义上的语言本身。家庭作业和居家式学习之间的区别是什么?跟孩子们谈论他们的学习任务和跟孩子们谈论他们的学业之间的区别是什么?一个孩子们的家庭作品展和旨在提高学习能力的展示两者之间的区别是什么?老师们可以在自己的学校尝试着实验这两类问题之间的不同————一定会得到不同的结论并效果显著。\par
     越来越多的数据证明:了解到我们的大脑是如何运作的并将这些成熟的理论运用到学校的个体教学中来可以使学校的教学效果有显著的提高。总结一下就是学习本身也是可以学习的。本单元的学习不仅帮助老师和学生深入了解大脑的运作机理,还提供了一个提高每一个人元认知的机会————这些是思考、反馈和提高学习能力的必要条件。 \par
 
\subsection{IPC课程的学习过程}
     整个IPC课程围绕一种理论过程展开,这种过程持续关注5到12岁的孩子学习的方式。学习到整个课程内容对孩子们来说是远远不够的,他们还应该认识到IPC这种教学方式的好处和重要性。 \par
     

\subsubsection{学习入口}
    这是整个IPC课程的开始环节,过程也很有意思和刺激。 \par
    学习入口的目的是为了让孩子们思考和商讨即将要学习的内容。

\subsubsection{获取知识}
    本环节帮助老师们了解孩子们就即将要学习的主题已经掌握了些什么。然后在孩子们想要学习的东西基础上为他们个性化定制课程,从而对原来的课程进行适当的删减。这样做加强了学生旧知识点和即将学习的知识点之间的联系,从而然然孩子们取得学习的主动权。\par

\subsubsection{主题阐述}
    本环节在整个课程开始之前为老师和学生提供了单元的学习概要,这样做的好处是可以促进学习主题和概念之间的联系。\par

\subsubsection{探究活动}
    每一个主题学习都会有探究活动,该环节被设计的目的是来确保孩子们能够以正确的方式来获得信息,可以尝试多种学习方法比如角色游戏、数字学习、图书馆查阅等。这些方法中有一些方法之间是有联系的,还有一些方法被设计出来发展学习个体的探究能力和恢复能力,进而促进和改善IPC的个人目标。

\subsubsection{记录活动}
    该环节帮助孩子们加工和展示他们在探究活动中获得的各种信息和知识。获取这些知识的方式旨在锻炼他们的各种能力和兴趣,从而促进其记录学习的能力。记录学习的方式可以有很多种,比如说数字设备记录,戏剧形式、作曲、地图、图片、实验、艺术作品等等。

\subsubsection{学习出口}
    本环节结束整个章节的学习。最后让孩子们在他们以前的展示中画上新学过的知识点。这种方法有助于提醒孩子们联系起已经完成的主题之间的知识点。并提供了时间和机会来对已经学习的内容作出自己的理解,然后以小组为单位进行思考。学习出口提供了和父母接触的绝好机会,也可以邀请他们一起庆祝本单元的顺利结束。
