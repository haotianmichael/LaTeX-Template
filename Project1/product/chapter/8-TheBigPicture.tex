\chapter{知识图谱}
    IPC为里程碑1,2,3写了三个脑波单元(学习的艺术)。每一个单元帮助老师和学生了解学习是什么,学习意味着什么,学习有什么内容可谈论,如何评价学习成果,在课堂上不同类型的学习会有什么样的实践结果,如果将学习地点变成户外会怎样呢等……
\par
    我们建议在每个学年开始时教授这些单元的各个方面,让孩子们学会学习的思维模式。\par
\subsection{IPC 自我检查}
    IPC自我检查文档和进程资料对所有的会员学校都是开放的,可以在我们的会员区进行下载。它通过设立与学习者,教师,领导者和社区相关的初级,发展和掌握水平来指导学校完成IPC实施和发展的9个关键领域,即“底线九”。\par
    在接下来的内容里自我检查会跟着涉及到很多的细节(比如:国际意识、个人目标、班级活动、评价、不同类型的学习、关于学习的最新的研究等)并且就如何和不同的利益相关者共同发展这些领域给出了详细的指导。

\subsection{关于学习}
    ‘我们学校是IPC学校’和‘我们使用IPC来帮助提高孩子们的学习水平’这两者之间有什么去吗?抑或是我们有一些钻牛角了。我们认为这两种称述之间的不同本质上在于IPC到底是什么。 \par
     经验告诉我们那些最成功的学校永远都将提高学习质量深深地嵌套在自己的教学体系中————所以这些学校中的孩子在学术、私人和国际化领域学习效果都是最好的。\par
     但是什么才是学习、我们如何才能真正掌握孩子们是否、在什么时候、用哪种方法才能从学习中获取知识?本单元的学习同时帮助老师和学生去深入思考这个问题,然后总结出一套学习语言供所有利益相关者使用。  \par
     这套学习语言要超过语义上的语言本身。家庭作业和居家式学习之间的区别是什么?跟孩子们谈论他们的学习任务和跟孩子们谈论他们的学业之间的区别是什么?一个孩子们的家庭作品展和旨在提高学习能力的展示两者之间的区别是什么?老师们可以在自己的学校尝试着实验这两类问题之间的不同————一定会得到不同的结论并效果显著。\par
     越来越多的数据证明:了解到我们的大脑是如何运作的并将这些成熟的理论运用到学校的个体教学中来可以使学校的教学效果有显著的提高。总结一下就是学习本身也是可以学习的。本单元的学习不仅帮助老师和学生深入了解大脑的运作机理,还提供了一个提高每一个人元认知的机会————这些是思考、反馈和提高学习能力的必要条件。 \par
 
\subsection{IPC课程的学习过程}
     整个IPC课程围绕一种理论过程展开,这种过程持续关注5到12岁的孩子学习的方式。学习到整个课程内容对孩子们来说是远远不够的,他们还应该认识到IPC这种教学方式的好处和重要性。 \par
     

\subsubsection{学习入口}
    这是整个IPC课程的开始环节,过程也很有意思和刺激。 \par
    学习入口的目的是为了让孩子们思考和商讨即将要学习的内容。

\subsubsection{获取知识}
    本环节帮助老师们了解孩子们就即将要学习的主题已经掌握了些什么。然后在孩子们想要学习的东西基础上为他们个性化定制课程,从而对原来的课程进行适当的删减。这样做加强了学生旧知识点和即将学习的知识点之间的联系,从而然然孩子们取得学习的主动权。\par

\subsubsection{主题阐述}
    本环节在整个课程开始之前为老师和学生提供了单元的学习概要,这样做的好处是可以促进学习主题和概念之间的联系。\par

\subsubsection{探究活动}
    每一个主题学习都会有探究活动,该环节被设计的目的是来确保孩子们能够以正确的方式来获得信息,可以尝试多种学习方法比如角色游戏、数字学习、图书馆查阅等。这些方法中有一些方法之间是有联系的,还有一些方法被设计出来发展学习个体的探究能力和恢复能力,进而促进和改善IPC的个人目标。

\subsubsection{记录活动}
    该环节帮助孩子们加工和展示他们在探究活动中获得的各种信息和知识。获取这些知识的方式旨在锻炼他们的各种能力和兴趣,从而促进其记录学习的能力。记录学习的方式可以有很多种,比如说数字设备记录,戏剧形式、作曲、地图、图片、实验、艺术作品等等。

\subsubsection{学习出口}
    本环节结束整个章节的学习。最后让孩子们在他们以前的展示中画上新学过的知识点。这种方法有助于提醒孩子们联系起已经完成的主题之间的知识点。并提供了时间和机会来对已经学习的内容作出自己的理解,然后以小组为单位进行思考。学习出口提供了和父母接触的绝好机会,也可以邀请他们一起庆祝本单元的顺利结束。 \par

    IPC自我检查标准7:\textbf{IPC的学习过程}为IPC的教学结构提供了更详细的指导。


\subsection{IPC的学习领域:学科、个人和国际学习}
   IPC学习单元结合了三个领域:学科、个人和国际。

\subsubsection{学科学习}
    IPC课程是严格按照选择好的主题来组织的。主题(比如主题‘巧克力’)反映了一定年龄阶段孩子的兴趣和关注点。这样的好处在于可以直接促进和驱使他们主动式的学习并发现更多有关的内容。在这些主题中,孩子们可以通过不同的学科视野来学习不同的内容,这些学科都包括:科学、历史、地理、音乐、艺术、信息和通信技术\footnote{ICT是nformation and communications technology.的简写}、科技、体育、社会学以及国际学科等。实际上在IPC,孩子们不需要深入的学习地理这门学科。我们聚焦于不同主题在地理方面的知识点比如‘奥运会’、‘假期’、‘健康’。用这种方式组织学习也可以让孩子们初步了解学科之间的相互交叉和相互独立。从而对即将学习的知识点有一个大的图谱,在图谱中进行不同知识点之间的连接,不同知识点之间的交叉并且从不同的角度来商讨所学习的内容。在线规划部分(见会员区)帮助学校计划需要执行的教学单元并并跟踪单元内每个科目学习目标的覆盖范围,这样可以确保课程的广度和平衡性。\par
    IPC自我检查标准8:\textbf{通过相互独立但是又相互依存的主题进行主题实施}对主题学习提供了更多的指导。

\subsubsection{个人学习}  
    IPC的个人学习目标指的是那些我们认为21世纪对孩子来说最重要的学习品质和学习倾向。\par
    IPC总共有8种个人学习目标:探究,适应力,道德,沟通,体贴,合作,尊重和适应性。体验和练习这些特定的学习目标已经被固定在每一单元的学习任务中并在每一个单元学习结束的时候提醒老师们进一步优化主题。课堂教学和学校的各个方面都应该体现出对这些品质不断追求的努力。将学校的课堂教学和这些个人目标联系起来有助于学校建立教学目标上的头统一性并帮助他们更容易向别人解释学校的整个教学计划。   \par
    IPC 自我检查标准2:\textbf{关于我们正在帮助发展的儿童种类的共同愿景}为如何进一步发展和适应IPC个人目标提供了更详细的指导。    

\subsubsection{国际学习}
    在帮助孩子们拥有逐渐成熟的国家,国际,全球和跨文化视角、发展一种国家化意识方面,IPC具有独到的优势来定义国际化学习的目标。每一个IPC单元都有涉及到国际化,在不同的学科之间以学习为主的活动除了鼓励做更多的全球视野的活动之外,还帮助孩子们发展一种国际意识并对自己、对自己生活的社区、世界有了更深的认识。除此之外,每一个IPC主题不仅鼓励学校从两个不同的视角:居住国家(学校所在的国家)和祖国(孩子们的祖国)来探究单元内容,还将包含国家化内容的主题作为一个学科。有时候居住国家和祖国是一个国家,这个时候学校就应该找到至少一个国家来探究该单元的学习内容。\par
    IPC自我检查标准4:\textbf{国际意识}为学校的国际化教育提供了更多详细的指导。

\subsection{IPC的学习类型:知识、技巧和理解}


\subsubsection{知识}


\subsubsection{技巧}

\subsubsection{理解}




