\chapter{学习出口}
   解释一下学习出口是整个旅途中,孩子总结自己所学的知识点,和别人分享,庆祝学习完成的时候。\par
   告诉同学们学习出口的目的就是要将学校变成更加积极的学习环境。\par
   以班级为单位,首先对整个学习过程中学到的知识点进行思维导图式的汇总。然后遍历你的学校,思考一下你想要创建或者提高的学习类型和学习模式。\par
   一起合作,思考一下那些在时间上(或者空间上)最容易实现的例子。比如说:\par

   \begin{itemize}
     \item \textbf{早餐俱乐部}--确保那些没机会吃早餐的孩子一定吃早餐
     \item \textbf{多喝水}--教室有饮水机吗 
     \item \textbf{健康的零食}--你们学校的超市都卖什么类型的零食?在早餐的时候也会卖吗?思考一下什么样的零食应该被提倡? 
     \item  \textbf{放松地区}--学校有专门的放松的地区吗?或许可以考虑一下,如果学校有一个高档的花园需要被改造,那考虑一下可以改造成为一个‘放松区域’。
     \item  \textbf{图书馆}--你们学校的图书馆区域有没有改善来帮助学生更好的学习?这里也许可以添加与IPC学习之旅相关的显示和对知识,技能和理解的解释。 还可以提供当地的信息以及孩子们与家人一起探访的地方。
      \item  \textbf{技能展示}--同可以自行组织“放学俱乐部”或者“月俱乐部“来教其他同学自己精通的技能。
      \item \textbf{大脑锻炼}--孩子们可以通过每天早上在大厅里进行锻炼来分享他们所学到的一些练习 - 或者制作练习单,并将他们的想法呈现给其他课程以融入他们的日常生活中。
      \item  \textbf{一起学习}--你的学生是否知道其他班级的孩子在学习什么?老师可以创建一个公共区域来传达每个班级正在研究的主题单元。 这样就会有机会在中间建立联系并分享这种学习,以便于学校里的每个人都参与支持。
    \end{itemize} 
   
   一起合作思考那些在时间上或者空间上最容易实现的并且同学已经实现的事情。然后将其付诸实践。你可以将这些过程通过视频的形式或者图片的形式记录在文档中,然后分享和评价这些孩子们的成功经历。\par
   \begin{note}
     在IPC会员区有很多的家长手册。这些可以在活动期间提供给父母并解释其目的。这个时候孩子们应该已经精通知识、技能和理解之前的各种知识点。所以他们应该使用他们学习的例子来解释这些问题。
    \end{note} 
   IPC学习社区会很高兴看你们的学习案例,不管是那个主题、在学习过程中的哪一个阶段。吐过你有任何想要分享的图片或者视频故事,请参考一下的\href{http://www.facebook.com/InternationalPrimaryCurriculum}{反馈链接}.
   
