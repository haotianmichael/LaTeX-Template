\chapter{多视角}
    

\section{探究活动}
    展示一张有IPC十项课题的图片,每一个课题都以拼图的形式展示出来,比如说:\par
    请学生来对这个图片展示作出评价,就图片的拼图呈现方式,他们有什么想法。每一位同学最后都应该理解:每一个课题都有属于自己的不同视角和内容,因此每一个课题都是相互独立的。但是这些课题又相互之间影响和不断的呼应,他们连接在一起最后呈现出一幅本单元的‘知识图谱’。从这个角度来说,这几个课题之间又是连接在一起的。\par
    为了将这些课题付诸实践,告诉学生们进行分组练习。每一个小组探究不同的课题。在任务结束的时候,每一个小组在班级上展示他们所探究的成果,理论上每一个小组成员就本小组探究的主题和角度都应该更宽阔和更深入一些。\par
    接下的这个任务,选择一个建筑,在教室附近或者周边的地方。最理想的地方应该是那些有历史意义的并对公众开放的地方,可以使一座建筑、一片遗址、一处古战场等。试着选择一块地方,允许孩子们进行自己历史和地理课题研究的地方(这样做可以刺激孩子们探究的热情)。详细见下面:    \par
    给学生们看这个地方的建筑分布图。然后将孩子们分成若干组,每一个组分配到不同的任务。告诉孩子们他们将要‘通过课堂上学到的镜头’探究这个地方。(团队可以拥有一个标有主题标签的放大镜作为小组的吉祥物)。和全班的嘘声一起讨论一下每一个小组收集到一些信息。每一个小组成员都应该有自己想成为的目标,去帮助他们更专注于本小组的主题。\par
    比如说:\par
    \begin{itemize}
      \item 历史学家---这个地方是什么时候建立的,它的重要性及其原因。发生在这个地方的历史故事和壮举。\textbf{记录目标:}历史大事件表或者时间表;
      \item 地理学家---这个地方的地理位置,以及当时创建的时候为什么要选择到这个地方建造。该地区的物理和人文地理因素都有哪些。\textbf{记录目标}一张给游客看的地图;
      \item  艺术家---关于这个地方的艺术品或者和该地区有关的历史人物的肖像画。他们都是谁?我们从这些肖像画中可以体会到什么?\textbf{记录目标:}一个口头或者书面的报告;
      \item 科学家---该地区的体系结构和布局。\textbf{记录目标:}  一个建筑模型;
    \end{itemize}  
      
    \par
    根据你选择的探究地区,一定还可以涉及到更多其他可以探究的主题比如社会学(有多少种服务对象)和科技领域(研究和发现之间的关系)等。学生中也可以有多个小组做相同的主题的,这样也可以将收集起来的资料联合起来互相帮助,共同进行记录活动。\par
    事先让学生使用一切手头可以使用的查询工具比如书籍、网络、地图、游客手册、图片等来找到更多的和探究地相关的信息--当然主要精力还是要专注在自己探究主题。然后带上孩子去探究地,进行实地探究。当开始进行任务的时候,孩子们可以使用各种方式来记录自己的活动过程比如照相、视频(如果可以的话)、甚至画旅行地图等。重点关注与其探究的课题相关的细节。\par
    
\section{记录活动}
    回到课堂上之后,成员就用自己收集和探究好的资料帮助自己的小组制作时间线、旅行者手册、模型、画展等。在过程中小组之间也是可以进行互相沟通。并有机会讨论和自己小组的作品和心得。举个例子:历史学家发现的一些事实或许对地理学家制作地图的时候有帮助。艺术家们可以就如何阐述时间线给出自己的历史观点等等。\par
    在任务节结束的时候,给每一个小组时间来展示自己的作品,并谈谈自己小组的心得。一定要通过特定的“课堂镜头”来研究每个学科领域的重要性以及我们可以通过研究得到的具体事物。\par
    每一个主题又提供了多少新的视角和信息?将孩子们的观点和发现建立联系,帮助他们阐述和发现主题之间是有联系的。就像开始的拼图一样,蜘蛛网一样。\par

\section{个人目标}
    \begin{itemize}
      \item 沟通
      \item 合作
      \item 探究
      \item 深思 
    \end{itemize}
