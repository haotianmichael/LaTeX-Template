\chapter{获取知识}
     回顾获取知识的目的。这对每一个学习了本单元内容的人来说是表述自己所学的好机会。这样也便于探索新的知识点并将这些新的知识点和我们已经掌握的知识点连接(或者说是我们认为我们知道的)。通过这种学习方式,意味着老师们可以为学生们规划和提供主题范围之内更有意思的学习内容。 \par
     如果孩子们之前有过获取知识的经验,那就应该多花时间回顾那些不同的方式来分享和捕捉过去的知识,比如通过分组、班级讨论、知识图谱、图片文字等等。像知识图谱等类似的视觉方法能够帮助孩子们将整个单元所学的知识点连接起来并加强大脑寻求模式这一自然倾向。班级讨论那些最受欢迎和最高效的方式,你或许能够根据孩子们的建议来调整下面的内容。\par
     说明学习有三个阶段:知识、技巧和理解(一一展示这三个阶段)。孩子们或许已经熟悉了这些内容(参见这章的知识图谱和脑波1单元)。\par
     两个人或者以组为单元,让孩子们各自拿出一张纸。然后使用他们认为有助于表达或者解释的文字或者图片来把下面的内容写在纸上。邀请每一个小组来展示他们的作品。然后鼓励孩子们用一句话来描述所表达的。
     \begin{itemize}
       \item 知识是……
       \item 技巧是……
       \item 理解是…… 
     \end{itemize}  
      然后全班讨论。如果孩子们之前在上一单元就有过这种任务的经历,那老师手中一定有孩子们的上一次的描述内容。这时候拿出来进行前后对比————然后看看相比于上一次的描述,这一次的描述是不是更加全面。随着不断的学习,这时候应该是全面的。\par
      在同一个小组内,让孩子们记录下三个阶段————知识、技巧、理解最好的掌握方式。他们应该能够按照自己的经验、自己在学习入口活动中参观不同的班级时候的所见所闻来完成这一切。这些经验和所见所闻或许在他们定义这些阶段的早期反应中也起一定的作用。\par
      全班讨论这些方式,然后在每一个阶段的展示中记录这些方法。在整个单元的学习过程中这些都可以拿来参考,在未来的主题学习中也可以作为资料来使用。这些方法包括: \par
      \begin{itemize}
      \item \textbf{知识} -网络、书籍、采访、谈话、幻灯片、DVD、报纸、人工制品、文档、作品等
      \item \textbf{技巧} -实战、练习、建模、评价准则、车间、运动俱乐部、教程等
      \item \textbf{理解} -反思、辩论、换位思考、教授其他人、解决问题、挑战、出口、庆祝会等
      \end{itemize}  
\par
      思考为什么这三个阶段对学习那么重要————在为未来的学习过程中知识和技巧如何共同促进孩子们理解力的进一步的发展(具体细节参见这三个阶段的知识图谱)。\par
      结束部分。给每一个孩子机会来提出并记录自己关于学习的问题。这些最后也可以保留在展示上以后的主题学习也可以参考使用。孩子们也可以在他们考虑学习的其他问题时添加到这个显示中。\par
      让展示中有获取知识的部分是学习过程中很有用的工具。因为随着时间的不断推移,孩子们可以经常思考自己所学并将一些零碎的知识点拼凑起来。这是一种叫做满思考的学习方法(具体见知识图谱中元认知部分)。

      
     
