\chapter{去大脑内部看看}
    
\section{探究活动}
    以一场简单的‘连接’游戏开始这个任务学习。首先你需要一个大的毛线团,孩子们围成一个圈,将毛线的末端系在你的手腕上。\pat
    第一、每一个人都谈谈自己(比如自己最喜欢的颜色)。如果谈论内容有相似点的同学(比如喜欢相同的颜色),将手举起来。从这些孩子中挑一个,然后让他们分享一个关于自己的事情(如果他们卡住,提示可以使:比如喜欢吃的食物、爱好、娱乐活动、有几个兄弟姐妹、眼睛颜色、头发颜色、养的什么宠物、住在哪里、假期一般喜欢去哪里、最喜欢的电视节目等)。说完之后其余的孩子中如果有相同点的举手,说话的孩子选择其中的一个孩子并将毛线团丢给他,但是自己要留着线头以便两人之间靠毛线连接。随着这个游戏持续的时间越长,你就会看到一个很大的蜘蛛网一样的网络,最后如果可能的话,每一个孩子都有可能是一个节点。\par
    在课堂上讨论这个靠相同点连接起来的网。学生们觉得这样的网足够密集和强大吗?让每一个人都将自己手中的线头拉紧,助教拿出一个球放在上面,这时候会发现这个球可以悬空放在网上。\par
    让其中一个孩子将手中的线头节点放开,然后一个接一个的都放手。过程中该网络会发生什么?随着连接越来越少,整个网络会开始塌陷最终变得很虚软。\par
    课堂上告诉孩子们,我们的大脑也是以类似的机制工作的。我们的大脑中充满了叫做神经元的细胞,我们在学习了新的东西的时候,神经元之间就会建立连接。\par
    不仅仅会促进神经元细胞之间的连接,如果我们不断练习某项技能,我们的学习能力也会得到提高(参考前一项任务学习)。因为和这项技能连接的神经元细胞使用的频率会更快。这样神经元之间的结合就是更加结实,就像在游戏中做的一样。这也是为什么我们在不断练习的时候学习能力会不断加强的原因所在,也是为什么当我们遇到困难的时候不能随便放弃的原因之一。如果我们不加练习,这些神经元之间的连接会变的很弱甚至断掉。当选择放弃的时候,网络就会变弱。\par
    制作一些卡片,在课堂上使用。卡片上有关于大脑的对或者错的陈述。向学生展示这些卡片,一起讨论这些卡片的内容正确性,最后给出正确的答案。详细的关于大脑的细节见这些网址:\par
    \begin{itemize}
      \item 科学博物馆的链接有很多大脑有关的\href{http://www.sciencemuseum.org.uk/whoami/findoutmore/yourbrain.aspx}{部分}
      \item \href{http://www.theschoolrun.com/homework-help/the-human-brain-andnervous-system}{School RUN}进行了很多关于大脑和神经系统的研究
    \end{itemize}  
    接下来在课堂中展示关于大脑、神经系统、和神经细胞之间传输的电信号的图片。\par
    问学生在学习大脑的过程中是否有什么地方对他们带来了很震撼的体验。学习过程中有哪些是以前不知道的点?这时候也是重温知识收获环节和期间提出的问题的最好时机。

\section{记录活动}
    以小组为单位,学生可以想象自己刚刚从一场和大脑有关的奇妙之旅中回来。每一个小组可以做一个小展示或者写一篇关于这次奇妙之旅的记录文章。\par
    为了更好的完成这个任务,老师们可以提供一些资源:比如说书籍、图片和DVDS……供学生们阅读。这也是一个探究描述性词汇的绝好机会。Google一下就会发现很多关于神经系统的惊人的图片。这些图片也可以作为美术课的素菜,学生可以挑战一下自己的多彩神经系统。在任务结束的时候,孩子们也可以展示这些作品。

\section{个人目标}
    \begin{itemize}
      \item 适应
      \item 沟通
      \item 合作  
      \item 尊重
      \item 深思 
    \end{itemize}  

