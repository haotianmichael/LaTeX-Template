\chapter{保护我们的大脑}

\section{探究活动}
   回顾一下积极情绪为什么如此重要。在每一个小组中列出学生们认为能让大脑保持健康和警觉的所有因素。如果学生学习完成了里程碑第一章脑波单元,那他们一定想起来一些探究结果。老师手里还有他们的大脑日记,这时候可以拿给他们进行参考。\par
   讨论一下同学的观点。如果条件允许,可以看一下这个视频\href{http://www.bbc.co.uk/scotland/brainsmart/brain/#bb-emp}{BBC Brain Smart网站上有一个动画视频,其中包含一系列用于照顾大脑的重要提示。}\par
   让同学列出在视频中记住的所有建议————健康饮食、锻炼、水、睡眠等。谈论一下为什么这些都重要。并指出作为学生可以怎么做来帮助“改善自己的大脑”。\par
   其中的一些建议或许已经是学校的日常规范————起床活动、早餐等。或许老师还可以注意新的方法。\par
   参考网站:
   \begin{itemize}
     \item \href{http://www.gonoodle.com/}{gonoodle.com}提供免费的在线短期锻炼和放松程序库,由丰富多彩的动画人物组成。 如果您希望将交互式白板融入您的日常工作并真正吸引孩子们,这是理想的选择。
     \item \href{http://www.wakeupshakeup.com/}{wakeupshakeup.com} 醒来! 摇了摇! 为课堂练习提供自己的活动课程DVD和音乐CD。
     \item \href{http://www.youtube.com/watch?v=O5ChXC-rHLE}{youtubu.con} YouTube举办了3分钟的运动例行演示,以唤醒大脑。
     \item \href{http://www.bbc.co.uk/scotland/brainsmart/brain/}{bbc.com}BBC Brain Smart网站上有一系列关于大脑的有用视频和文章。
     \item \href{http://www.shakeupyourwakeup.com/}{shakeuo.com提供健康的早餐提示,膳食计划和免费下载资源。     
   \end{itemize}  
   告诉同学他们将会去给学校的其他班级教授自己的学到的关于大脑的理论。这次授课应该如何进行,如何记录,如何和别人分享呢?回顾一下同学们前几节任务中的观察和观点来帮助你。

\section{记录活动}
    同学挑战制作一个学习资源来教其他班级的学生来学习大脑。回顾一下自己在前几节中学习的关于大脑的知识点。然后从研究部分开始探究孩子们的观点————这些知识应该如何传递。比如:\par

    \begin{itemize}
      \item 护理指南小册子、文字和图片
      \item 卡片游戏,玩家要通过收集和饮食、锻炼等相关的卡片来连接属于他们自己的神经元
      \item 教学视频,以自己为特色或使用木偶和模型
    \end{itemize}  
    在任务结束的时候,小组开始展示各自的作品。比较那些不同的方法。评估一下每一组的方法在传递信息方面、观众参与度上的效果。

\section{个人目标}
    \begin{itemize}
      \item 适应力
      \item 交际
      \item 合作
      \item 体贴   
    \end{itemize}  
