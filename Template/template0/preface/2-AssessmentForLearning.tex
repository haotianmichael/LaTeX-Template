\chapter{学习评估}

Assessment for Learning
Are your children busy, or are they busy learning? This is the question that we need to be able to answer throughout each IPC unit – what improvements are being made to children’s learning as a result of studying this theme?
There are three areas of learning to reflect on, and three types of learning to assess.
The Three Areas of Learning: Academic, Personal and International
The three areas include academic, personal and international learning. To reflect on these, you will need access to the IPC Learning Goals for each subject (including International) and the IPC Personal Goals – a list of these can be found in Appendix A of the IPC Implementation File. You can also find a full list of IPC Learning Goals in the Assess section of the Members’ Lounge.
The Three Types of Learning: Knowledge, Skills and Understanding
The three types of learning include knowledge, skills and understanding. We believe that differentiating between knowledge, skills and understanding is crucial to the development of children’s learning. We also believe that knowledge, skills and understanding have their own distinct characteristics that impact on how each is planned for, learned, taught, assessed and reported on. The implications of these differences are therefore far-reaching and deserve proper consideration.
Knowledge refers to factual information. Knowledge is relatively straightforward to teach and assess (through quizzes, tests, multiple choice, etc.), even if it is not always that easy to recall. You can ask your children to research the knowledge they have to learn but you could also tell them the knowledge they need to know. Knowledge is continually changing and expanding – this is a challenge for schools that have to choose what knowledge children should know and learn in a restricted period of time.
The IPC does not provide examples of knowledge assessment (tests or exams) as the knowledge content of the curriculum can be adapted to any national curricula requirements.
Skills refer to things children are able to do. Skills have to be learned practically and need time to be practiced. The good news about skills is the more your practice, the better you get at them! Skills are also transferable and tend to be more stable than knowledge – this is true for almost all school subjects.
The IPC supports skills tracking and assessment through the IPC Assessment for Learning Programme. This programme includes Teachers’ Rubrics, Children’s Rubrics and Learning Advice.
Understanding refers to the development or ‘grasping’ of conceptual ideas, the ‘lightbulb’ moment that we all strive for. Understanding is always developing.
The IPC units can’t assess understanding for you, but they do allow you to provide a whole range of different experiences through which children’s understandings can deepen.
(Please note: as well as the IPC Assessment for Learning Programme, we also offer an online Assessment Tracking Tool, developed in partnership with Classroom Monitor. Please email members@fieldworkeducation.com for more information on how to sign up to this tool.)
