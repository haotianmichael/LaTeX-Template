\chapter{学习成果}

Divide the class into groups. Provide the groups with three large sheets of paper. Ask them to write ‘School’ in the centre of the first sheet of paper. Explain that they are going to imagine they are showing a best friend around the school for the first time. What things would they show them? Ask the children to discuss this in their groups and make a list of the human features (buildings, playground, roads etc.) and physical/natural features (trees, ponds, flower beds etc.) of their school environment.
On the second sheet of paper, ask the children to write the words ‘Where I live’. Ask them to imagine that they are showing a best friend around where they live for the first time. What things would they show them? Ask the children to mindmap their thoughts and ideas.
Some children might start by thinking about their home or the town/village/city where they live. Prompt the children to expand each idea and mindmap the locations and features of each environment. For example, children might list the rooms of their house or some of the places in the local area that are important to them.
Finally, ask the children to write ‘My country’ on their third sheet of paper. Invite them to consider their host (or home) country and how they would describe it to a friend. Do they know the names of any towns and cities? What other features might be important – such as rivers, ports, famous landmarks and so on. Again, encourage the children to think about the human and physical features of their country.
Afterwards, view your school environment using Google Earth (google.com/earth/). Look closely at the aerial maps to identify the features that the children listed. Begin with your school, then expand outwards (by using the zoom tools) to view the local area, then the whole country.
Locate any of the towns, cities and other landmarks that the children know about. Use question prompts and directional language to encourage the children to explore the world map further. For example:
