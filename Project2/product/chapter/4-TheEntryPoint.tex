\chapter{学习入口}
    让同学们想一些自己比较擅长的事情。完全可以是课堂之外的事情。如果有同学提到了比如艺术、体育、音乐、电脑游戏这些外延比较宽泛的领域,让他们描述的更清楚一些。因为可能需要他们将自己的技能和知识教给别人。\par
    给每一个同学一张卡片,每一位同学可以在上面写上他们的名字和他们能够教给别人的东西比如:"40米开外的铅球扔法"、“站在别人的头上”、“同时玩4个球”、“如何同时快速得出两个数的乘积“……或许同学们的想法会更加奇特和有意思,老师在筛选的时候需要保证最基本的安全。\par
    告诉同学们他们即将成为别人的老师,将自己的高超技术传授给别人。将同学两人分组,这样每一位同学既是老师又是同学。例如,老师可以从帽子中抽出卡片,将它们显示在墙上并邀请同学们随机选择。\par
    首先留时间给同学去思考他们已经学习过的关于快速学习和快速教学的知识,然后收集每个人的观点,以这些为基础来规划教学。可以从这几个方面作出提示:\par
    \begin{itemize}
      \item 和教学内容相关的知识、技巧和理解。应该如何具体辨别和掌握?
      \item 老师的各种教学方法————老师会给学生提供什么阅读材料(比如帮助手册)、提供什么示范(建模)、会有什么具体的课堂展示(比如PPT)或者以上三者都有?
      \item 对学习者有利的事情————他们使用的语言、如何调整出他们的积极心态?这些和我们已经学过的关于学习的知识有什么联系?
      \item  学习评估————学习者如何知道他们是否进步?我们如何监督他们的进度并根据实际情况提供帮助?学成的标准具体会是什么?\footnote{这也是任务的一部分,学生们必须得自行设计自己的评估准则。不过学生可以参考他们已经很熟悉的IPC学习计划评估。如果学校并没有使用这套计划,那学生可以考虑自行简化和修改学校的学习评估系统。IPC评估准则一共有三个阶段:初学、一般和精通。希望这些可以帮助学生自行设计出自己的评估准则。(IPC准则可以再哎会员区下载)} 
    \end{itemize}
    \par
    固定时间让学生们设计和准备他们即将需要教授的内容。并且需要考虑到他们上课时候需要用到的道具。\par
    老师需要将学习入口上课时间划分为两部分。这样第一部分结束之后学生还可以有足够的时间去准备和配备他们上课需要的装备、材料和教学资源。他们的学习过程和教学过程也会非常顺利。\par
    一旦学生准备好,两人分组开始第一部分:\par
    给同学们最多1小时时间来实施完成第一部分的教学和学习。然后两人身份交换,老师成为学生;学生成为老师。给另外一小时进行第二部分的学习和教学。在每一部分的过程中鼓励孩子们互相监督和评价学习进度——在每部分结束的时候,给双方一次机会一起完美结束。\par
    同学可能会使用自己的视角来进行监督。如果你正在使用IPC学习计划评估,那么三个阶段(初学、一般、精通)可以比做成一座高山,到达山顶就是精通。又可以看成是还在生长的大树。老师们甚至可以做一个全班同学共同的监督工具。每一位同学可以将自己的图片放到相应的阶段上。每一个人都应该确保学习并不是竞争,每一次评估将会是一次帮助我们了解我们的学习过程,我们还可以上升的领域、我们需要提高的部分的一次机会。\par
    全班一起讨论过程中到底发生了什么?你或许想画一张思维导图来记录这一切比如:\par
    \begin{itemize}
      \item 学习起来很容易因为……
      \item 学习起来很吃力因为……
      \item 教起来很容易因为……
      \item 教起来很吃力因为……
    \end{itemize} 
    问同学,如果他们即将有另一次机会来做一遍。他们希望哪里被改善来提高教学效果。\par
    学习入口的结束应该是做出一个”如何成为一个更好的老师的标准“或者”如何成为一个更好的学生准则”。然后将这两份报告贴在展示区,在接下来的一年内都可作为参考。\par
    学生们应该对学习入口的内容非常了解。但是如果你的学生是新来的,对IPC不是很了解,老师需要复习一下相关内容。每一个IPC学习主题都会有一个学习入口,学习入口的目的主要是让学习者对即将学习的主题提起兴趣,并且有一些这方面的思考。老师可以问一下学生们学习入口到底对自身的学习过程有什么正面作用,甚至有什么进一步的帮助。\par
    
    
