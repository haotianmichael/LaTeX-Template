\chapter{获取知识}
    如果可能的话,一起回顾一下获取知识的目的。知识获取其实是一次机会让每个同学来分享他们关于即将开始的单元已经知道的事情。当然也是一个来探索关于我们耳熟能详的主题更多不为人所知的秘密的机会。\par
    上课开始的时候,在班上展示一些关于大脑和学习的陈述。你可以想这些陈述交给一个小组,或者有好几组的孩子围在一起来来玩。告诉孩子们这是一些关于大脑和学习的观点,在小组内一起尝试论证一下这些观点的正确性,讨论一下他们是否同意这些观点。并给出自己的理由和想法。可以使用便利贴写下来。如果每一个小组轮流来一次这些观点,那么其实可以参考一下前几个小组关于该观点的看法,并按照情况将这些观点加到自己的陈述和想法中来。\par
    关于学习和大脑的观点可以包括以下几个方面;\par
    \begin{itemize}
      \item 环境(家庭和学校)直接影响大脑的学习效果。
      \item 你的大脑越大,你越聪明。 
      \item 区分以下左脑思考者和右脑思考者。左脑思考者更加理性和逻辑化一些。而右脑思考者更加感性和情绪化一些。
      \item 学习者的类型是天生的。
      \item 用古典音乐作为胎教音乐会让小孩很聪明。
    \end{itemize}  
    结束之后,全班一起讨论一下这些观点。发挥同学的思考和想象能力。他们认为这些观点中有哪些是正确的?在他们自身的个人经验中有哪些可以直接反驳这些观点?一起探索一下学习的过程,大脑是如何工作的。老师可以透露一下这些观点都是neurimyths\footote{无实际意思,下文有解释}————都是一些被证伪的或者没有充分的证据证明的观点。\par
    关于更多的很流行的一些neurimyths。请参考一下几个链接:\par
    \begin{itemize}
      \item \href{http://www.brainfacts.org/neuromyths/}{brainfacts.com}该网址探索一些被证伪的观点及其原因。
      \item \href{http://www.oecd.org/edu/ceri/neuromyths.htm}{oecd.org}该网址提供了5个关于neuromyths的解释。 
      \item
    \end{itemize} 
    以小组为单位,每一位同岁都需要写下自认为关于大脑和学习正确的一些观点。他们应该使用他们在课堂上的经验来帮助自己。如果同学已经学习完成脑波上一单元的课程。那这也会是一个好机会来展示关于上一单元大脑和大脑的工作机制还记得多少。小组做课堂报告。\par
    花时间进行讨论,每一个小组都认为这些观点是正确的吗?有没有其他不同的观点?孩子们可以使用一些事实和例子来证明自己的观点。任何与本主题有关的词汇都应该被记录下来作为以后学习过程中参考的资料。\par
    老师最后提出这些观点都是不正确的。思考我们已经知道的关于大脑和学习的一些观点是否正确。为什么还有以下领域出于争论阶段?对老师和学生们来说,对这些知识保持最新的了解是不是很重要?\par
    老师也可以谈一下关于大脑成像技术\footnote{有很多种,比如MRI扫描}以及这些技术是如何帮助我们理解大脑的工作机制的。这些技术可以帮助人类检测癌症、帕金森症等病痛的端倪,也可以观察当我们思考的时候,学习的时候,理解的时候大脑中的具体活动。甚至我们的记忆是如何产生的。最后问一下孩子们关于这项技术他们认为在他们的有生之年可以看到什么样的进步?\par
    保留孩子们关于学习和大脑的观点并在本单元和之后的单元学习中适当予以展示,这样孩子们也会不断的明白自己的观点是否有所进步,之前观点是否正确。小组甚至可以修改自己的观点并提出新的观点。\par
    \begin{note}
      因为在任务6中检测IPC个人目标的时候,会使用到一部小说的人物作为同学学习的焦点。知识获取是一个好机会来提到你即将选择的小说以及这样做的原因。更多的信息可以看一下任务6和相关的语言艺术链接。
    \end{note}  
