\chapter{情绪和学习}


\section{探究活动}
   举办一次圆圈活动\footnote{全班围成一个圈的活动}.让同学看下面的话,请同学将自己的感觉和回应记录到一张卡片上。然后按照下面的两句话将卡片分成两份分别放到两个碗里,将碗放在圆圈的中央。然后同学一次取走一张卡片然后大声读出来。每一位同学用自己认为合适的方式将这些话组织起来。\par
   \begin{itemize}
     \item 当我……的时候我一般学习效果好(同学的回应包括:支持、兴奋、感兴趣等)
     \item 当我……的时候我一般学习效果不好(同学的回应包括:累、饿、冷、生气、担心等)
   \end{itemize}  
   和同学一起讨论一下不同的回应(老师也参与一些分享自己的回应)。如果可以的话,让孩子们把从这些回应中得出结论写下来。\par
   想同学介绍大脑中的一部分叫做’杏仁体‘区域。该区域是大脑中很多地方的交叉。所有的发送到大脑中其他区域的信息一般首先通过杏仁体。当我们感到害怕或者收到压力的时候,杏仁体中的神经元会迅速做出回应将我们转换为’战备‘状态(这是人类生来具备的一种生存能力,当人类还在狩猎时候,这种直觉给了我们生存下来的能力。)\par
   当人体的’战备‘状态被激活之后:\par
   \begin{itemize}
     \item 血液从身体的四肢冲出来保护主要器官(这就是我们的手冷的原因)
     \item 我们的心跳加快来获得更过的血液和氧气给肌肉来帮助肌肉运动
     \item 我们开始流汗紧接着身体发冷
     \item 大脑出于自动反映状态。这时候皮质(正常的反应和思考区域)几乎停滞,很难理性的思考和作出回应
   \end{itemize}  
   问一下同学是否有这种经历。当这种时候他们能够很好的控制自己的情绪和思维?是不是很难专注甚至保持冷静?\par
   探究一下可以帮助人们冷静下来的物理方法。当我们冷静下来之后,一般来说更容易思考和学习。'正念'的技巧通常很有帮助。这种方式鼓励人们对自己的意识有所关注————比如关注与自己的呼吸或者关注自己周边的环境比如当时环境的声音或者可以触摸的东西。这种方法可以训练我们的大脑寻找一丝的平静并且帮助大脑排除那些压力和焦虑的来源。\par
   老师介绍几种简单的呼吸练习给同学。比较好的是,这些方法老师都试验过并且觉的很有效果,这样的话同学也会更容易接受并认真对待这些练习。\par
   '正念'是一种纯意识的活动。所以老师可以带领同学进行很多的意识层面的活动练习。比如:给同学一个水果,然后让孩子们将所有的注意力放在这个水果上,包括这个水果的形状、味道以及所有的细节上,甚至是在咬的时候发出的声音。逐渐意识到’正念‘主要指的是将注意力主动放在我们周边的东西上,而不是将注意力放在我们平常习惯上应该放在的东西上。\par
   下面的网址提供了很好的背景信息和例子:  \par
   \begin{itemize}
     \item \href{http://www.parentscanada.com/school/tweens/teaching-your-tweenmindfulness}{parentcanada.com}ParentsCanada快速概述了正念及其在家庭和课堂中的使用。
     \item \href{http://mindfulnessinschools.org/mindfulness/}{mindfulnessinschools.com} 学校中的正念提供有关正念益处的信息。 该组织还在全球20多个国家提供课程和培训。
     \item \href{http://www.theguardian.com/education/teacher-blog/2013/jun/24/mindfulness-classroom-teaching-resource}{theguardian.com} Guardian Education网站提供了一系列有用的练习,可以在课堂上试用,帮助您的孩子介绍正念。
     \item \href{http://www.huffingtonpost.com/sarah-rudell-beach-/8-ways-to-teachmindfulness-to-kids_b_5611721.html}{huffingtonpost.com}赫芬顿邮报(Huffington Post)有一篇探索正念的短文,有八种简单的方法可以将它介绍给你的孩子。       
   \end{itemize}  




\section{记录活动}
    向同学介绍’神经生成‘的概念————大脑的成长过程。在脑科学领域最近的一项重大突破表明:大脑是可以有能力产生新的细胞的。其中一个和细胞生成有关的大脑区域就是海马体————一个和记忆有直接关系的领域。从这项发现可以得出更深远的结论:不断的练习、营养和减压都能帮助和促进老脑中这些细胞的生成。\par
    小组为单位或者全班一起思考:如何创造一种更加积极的更加有效果的学习环境,鼓励同学在思考的时候将自己的学习经历和刚刚学习的关于神经学的知识用上。 他们应该尝试其中的两三种建议并尝试运用一下这些建立————尝试一下这些建议是如何帮助促进自己的学习正向发展的。\par
    邀请小组来分享自己的观点。将这些观点列成一个表。全班一起决定哪一种建议可以尽可能快的实行。比如说:\par
    \begin{itemize}
      \item 奖励改进和努力的徽章
      \item 定期举办学习庆祝活动
      \item 在早上或者繁忙的学习之后放一些放松音乐 或者使用一些更加有活力的音乐来起床。
      \item 将植物引入教室,使环境更加宁静和放松,同时改善空气质量
      \item 将一个疲惫或杂乱的教室区域进行改造 
      \item 设置冥想/正念练习的时间(例如早上5分钟和/或午餐后)
      \item 每周有午餐时间或课后放松俱乐部,孩子们可以在那里练习冥想和瑜伽。 (可以邀请父母或当地社区的成员来领导会议。)
      \item 创建一个担心框(孩子们可以发布他们可能有的任何疑虑或担忧,可以通过一对一会话解决)
     \end{itemize} 
    老师或许很想继续探究正向思考的其他领域,比如锻炼和健康的饮食。(详细的键里程碑2脑波单元)\par
    实践孩子建议中的几个,并定期讨论来反思这些建议是如何帮助孩子们变得更加放松和成为更加有信息的学习者。\par
\section{个人目标}

      \begin{itemize}
        \item 沟通
        \item 合作
        \item 尊重 
        \item 深思
      \end{itemize}  
