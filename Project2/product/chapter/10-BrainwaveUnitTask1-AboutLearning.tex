\chapter{神经元和大脑}
   
\section{探究活动}
      \begin{note}
        如果同学已经学习上一章节脑波单元的学习,那他们应该已经熟悉了神经元和大脑的运作机制。这样的话同学们仅仅需要复习之前学习过的知识,然后将多余的探究任务作为家庭作业完成拿到班上和大家分享。        
      \end{note}  
      让同学在纸上写下一个单词,别让任何人看到这个单词。这个单词可以使一个名词,动词或者形容词。然后让一个同学拿着这个单词去找另一个同学,看看这两个同学的单词是否有什么相同点。如果有一些相同点的话,那这两个同学便可以组队去找第三个相同单词的同学。\par
      无论这条链在什么时候断掉(这种情况也可能在第一同学身上发生),选择另一个同学然后继续这种活动。看一看这条链到底可以有多长,时间可以持续多久。最后可能有的链只有一个人而有的链非常长。\par
      然后,让孩子们举起双手,伸开手掌心。用一张图片或者在白板上画的一张快照来向同学们引入神经元的概念。指出手掌就好像是神经元的主体,而手指就代表了树突\footnote{神经学专业术语}胳膊代表了轴突。\par
      介绍神经元的两项功能:第一,当被传入的信息刺激的时候,它们就会打开。一旦被刺激,神经元之间就会试图建立连接。如果两个神经元有相似之处,那么其中一个的轴突会发送信号,而另一个神经元的树突便试图接受这个信号。这是我们之前学到的。\par
      大脑会在我们经历和学习新东西的时候,创造各种各样的连接。当我们不断联系一项爱好或者技巧的时候,已经有连接的神经元变得更加坚固,同时也会产生更多的连接。大脑对这些不同的连接划分出出优先级以便于更高效的工作。\par
      让孩子们考虑一下这个过程和自己学习过程之间的联系。比方说我们对一项技巧联系的越久或者时间越长,我们对它的掌握程度就会越牢固和使用起来月高效————就像神经元之间会变的更加的高效和牢固一样。帮助孩子们理解大脑不是静止不动的, 而是不断的改变来应对我们正在经历和学习的新东西。这个过程被称为可塑性————通过加强神经网络来改变我们的大脑的行为。你可以将其比作是肌肉和健身的关系,我们锻炼的越多,肌肉越强壮。真实的大脑中充满了奥秘……这也是一个充满刺激的纪录片的名称————同学们被邀请去制作第一集。他们对自己演出的内容很负责任,因为这部纪录片将会帮助其他的孩子认识大脑会有多神秘。\par
      将同学分为若干组,提供一些资料来进行大脑内容的进一步探究。老师提供提示表,帮助孩子将注意力放在主要探究的内容上。这项提示表包括:\par
      \begin{itemize}
        \item 额叶
        \item 顶叶
        \item 颞叶
        \item 海马体
        \item 大脑  
        \item 小脑
        \item 脑干
        \item 下丘脑
        \item 神经元
      \end{itemize}  
      这是一些可能会用到的链接:\par
      \begin{itemize}
        \item \href{http://tinyurl.com/odp7ht3}{tinyurl.com}Brainline具有大脑的交互模型,解释了主要部分及其负责的内容。
        \item \href{http://easyscienceforkids.com/all-about-your-amazing-brain/}{easyscience.com}Easy Science for Kids提供大脑及其功能的有用概述,并附有图像和简短视频。
        \item \href{http://kidshealth.org/kid/htbw/brain.html}{kidhealth.com}kidhealth通过有用的图表和动画探索大脑和神经系统。
        \item \href{http://www.bbc.co.uk/scotland/brainsmart/brain/}{bbc.co.uk}  BBC Brain Smart网站上有一系列关于大脑的有用视频和文章。
        \item \hren{http://www.sciencemuseum.org.uk/whoami/findoutmore/yourbrain.aspx}{sciencemuseum.com}科学博物馆网站上有一个关于大脑的优秀部分,有视频,图像和许多“你知道吗”的事实。

      \end{itemize}  




\section{记录活动}
     根据他们的研究,学生们应该计划和准备自己的报告和展示。如果老师需要,展示被制作为视频作为学习出口庆祝的一部分。孩子们应该考虑如何让自己的学习成果更好的展示出来。老师可以根据时间的充裕与否,增加制作道具模型、服饰和其他可以让展示更加有趣和生动的视觉道具。老师们需要提示学生们:学生们会是老师,而观众会是学生,所以应该如何让自己的展示和教学更加使人印象深刻?老师或许需要看一下同学在学习入口时候的作品然后据此帮助孩子们设计他们展示。\par
     本章节以每一小组展示自己的内容作为结束。班级应该确定一个评价学习效果的评价体系。讨论一下孩子们关于大脑又学到那些新的知识。又或许这些新的东西又激起同学对大脑哪些方面的兴趣和问题。这些都可以加到班级的知识获取中,以便于作为每一单元的探究主题。\par
     老师也可以将视频展示给其他班级的同学,并从中得到反馈和孩子们一起分享。\par
\section{ICT链接}
    孩子们也可以使用视频制作软件比如\href{http://www.microsoft.com/}{WindowsMovieMaker}或者\href{http:}{PinnacleStudioUltimate}来剪辑自己的镜头并作出最终版本的展示视频。包括一个标题和信用序列。



\section{个人目标}
   \begin{itemize}
     \item 适应
     \item 沟通  
     \item 合作
     \item 探究
     \item 深思  
   \end{itemize}
