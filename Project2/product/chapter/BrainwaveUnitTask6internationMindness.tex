\chapter{国际意识}




\section{探究活动}
    在环节之前,请同学开始收集不同的国家的新闻故事(可以从报纸上,或者从网站上等)。尤其是一些很容易成为国际新闻的故事,比如自然灾害,贫困的灾害,战争,传染病等。\par
    老师可以参与到活动中来开始收集信息。在班上准备一个白板,同学可以将自己收集的新闻展示在白板上。\par
    \begin{itemize}
      对所选择的新闻类型始终保持警惕状态,因为有一些内容对某些孩子可能不是很适合,所以要展示进过筛选之后的新闻。老师挑选的新闻可能应该有助于为适合分享的新闻类型提供模板。
    \end{itemize}  
    下面的几个网址是选择世界新闻的理想方式:\par
    \begin{itemize}
      \item  \href{http://www.pitara.com/news/news_world.asp}{pitara.com}Pitara是一个简单的严肃的网站,提供来自世界各地的活动和问题的一口大小的故事。
      \item  \href{http://www.ourlittleearth.com/}{ourlittleearth.com}ourlittleearth提供来自世界各地的最新新闻。
      \item  \href{http://www.bbc.co.uk/newsround}{bbc.co.uk}BBC Newsround网站是世界新闻报道的良好来源。
    \end{itemize}  
    一起看一下这些新闻,然后简单的谈论一下同学们分享的这些新闻。在一张世界地图上高亮标出班上讨论过的地区和国家。在新闻中出现的领域是否有任何模式?是否有特定的热点可以识别?讨论一下这些标出的国家和城市,以及孩子们对这些地区有的一些知识。或许有的同学在假期旅游的时候去过这些国家和城市,又或者有亲戚住在这些国家和城市。已发现的任何事件是否会对我们的东道国/本国产生影响?\par
    \begin{note}
      在之后老师开始准备IPCMaking the news单元的时候\footnote{制作新闻单元},老师或许需要将这些新闻收集起来,为这个单元做准备。
    \end{note} 
    提示同学们思考为什么,作为一个学生和国际公民,对其他国家和社区有了解对我们的发展和未来来说是一件好事呢。同学可以首先对这个问题进行探究,然后将自己探究的结果分享给全班的同学。将这次的讨论和IPC国际化学习以及‘国际化意识’的概念联系起来,促进同学发展全球意识,更加认识自己,我们的社区,和我们所处的环境。\par

\section{记录活动}
     同学挑战完成创建一个属于自己的'国际化意识'的logo。它如何表达我们自己和不同社区,我们的学校,我们的东道国(学校所在的国家)以及这个世界?这时候同学可以回过头来参考一下探究活动中提高的一些问题。鼓励同学尽可能的体现出新型的设计思维,而且最重要的也是这个logo应该体现出‘国际化意识’的概念给他人。\par
     老师可以提供学校的物质资源,或者软件的使用权(如果可以的话),来帮助孩子们完成自己的设计。在设计结束的时候,举办一个展示,供孩子们谈论和展示自己的设计。老师可以举办投票,选出最受欢迎的设计,然后将这个设计作为你们班级国际化意识展示的核心部分。

\section{个人目标}

   \begin{itemize}
     \item 适应
     \item 沟通
     \item 探究
     \item 尊重  
     \item 深思  
   \end{itemize}  




