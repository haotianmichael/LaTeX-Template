\chapter{成长心态}

\section{探究活动}
    如果可以的话,在本课程开始之前,老师可以与班上的一些志愿者一起进行自己的迷你实验。这个实验是建立在美国一个以研究10岁左右孩童固定和成长心态的实验基础之上的。
    \begin{itemize}
      \item \href{http://www.youtube.com/watch?v=TTXrV0_3UjY}{youtube.com}由斯坦福大学的作者兼心理学教授Carol Dweck讲述的这段简短视频。 该视频探讨了对学习者心态的赞美。    
    \end{itemize}  
    
    \begin{note}
      一定要在安全模式下使用观看YouTube,滚动页面选择“安全”标签会展开信息栏。在“安全”信息栏点击“on”\footnote{英文:播放}。然后点击“safe”\footnote{英文:安全}  
    \end{note}  
    告诉学生这些活动是为了一个更大的活动做准备的。一个人为单位,为每一个同学准备一个简单的拼图来解决,这些拼图将使用即将镶嵌成的图案的形状。在活动开始之前,展示那个最终的图案,然后让孩子们使用手中的已经有的形状和图案来拼成提供的形状。\par
    如果可以的话,同学可以使用电脑软件来完成这些活动。\par
    \begin{itemize}
       \item \href{http://www.mathplayground.com/patternblocks.html}{mathplayground.com}Math Playground有一个形状编辑器,允许您从可用的形状中创建模式。
    \end{itemize}
    当孩子们完成了简单的拼图之后,请提供反馈。对于做的最好的那些孩子,夸奖他们是非常聪明的————“做的好,你一定得很聪明才能做出这个拼图”。对于那些做的一般的孩子,夸奖他们的努力————“做的好,你一定很努力才能实现这个拼图”。\par
    然后提供第二级简单的拼图,结束之后在进行夸奖。要么是智力,要么就是努力。\par
    接下来,给他们一个难度很高的拼图,然后让他们在部分形状的选择上也增加多样性。当他们完成这个难度的拼图之后,问他们接下来要进行哪种拼图————是前面的一种还是更加前面的\footnote{相比于高难度}的一种拼图。(不要说出’简单‘或者’复杂‘这样的词汇)。记录他们的选择。一般情况下,那些被表扬很努力的孩子更可能会选择比较难的那一个因为他们敢于尝试一般也不会注意到尝试之间的难度差距。而那些被表扬很聪明的孩子一般更可能选择比较简单的拼图,因为他们自认为很聪明但是其实接受不了不能做不成更难的拼图的那种挫败感。\par
    你可以记录整个实验并在上课的时候展示给全班同学。讨论一下结果以及看看孩子们根据反馈会有什么固定的反应。老师也可以提供视频然后对比一下两种反馈的不同结果,看看是否一样。(或许会又不一样的结果)然后让志愿者分享一下自己的经历。\par
    接着,给每一个小组一张两个孩子聊天的图片。并给每一个孩子发放一个对话框。提示他们思考:\par
    \begin{itemize}
      \item 孩子A对孩子B有什么看法,反过来孩子B对孩子A有什么看法?
      \item 在整个任务过程中谁的贡献和选择的难度最大?
      \item 老师还可以有哪些不同的做法,如果有的话?
      \item 两个孩子互相对对方都有哪些建议?
      \item 他们认为在这次任务中哪个孩子学习的最多,为什么?
      \item 谁本来可以做一些更难的活动但是最后没有做 ,为什么?
     \end{itemize} 
    同学可以在图片的周围用便利贴记录自己的想法和观点,或者使用一张不一样的思维导图。然后全班一起讨论一下方案,每一位同学对此有什么想法。整个过程是开放的,所以大家可以有不一样的观点。每一个提观点的同学最好可以解释一下为什么这样做比较好,并对其他孩子的建议作出评价。\par
    引导同学知道没有一个孩子是错误的,但是相比较而言,这个任务对于孩子B来说会更有价值一些因为孩子B一直很努力。任务本身很具有挑战性所以他们很容从中学到新的东西。孩子A可能并没有得到充分的测试,当任务结束之后其实能感觉到自己并没有学到太多的东西。或许下次孩子A可以选择更难的任务去完成。\par
    在白班上写下‘固定心态’和‘成长心态’。问同学自己的理解中这两个术语都有什么实际的意思。老师解释一下‘成长心态’的意思是相信我们总是能够变的更好。\par
    每一个人都是不同的————每个人都有对于技巧和能力不同的理解,但是如果我们总是不断的挑战自己和不断的将所学的东西应用到实际中,这样我们一定可以学到很多的东西(这时候老师可以主动回忆一下上一章学过的关于神经元的知识,关于重复和新的经历是如何帮助我们的大脑成长和进步的)。对于那些拥有着‘成长心态’的人来说接受挑战更是一次学习的机会,或许我们会犯很多的错甚至失败,但是从这些失败中也可以学到很多。\par
    ‘固定心态'更像是一种很死板的学习态度————相信每一个人的智力都是天生的,我们一生下来就会对某些事情很擅长或者不擅长。拥有’固定心态‘的人们更倾向于做自己领域内擅长的事情而不会触及自己不懂的新事物。因为他们相信这些新的事情不是为他们’生下来就准备的‘。在方案中哪一种心态会更适合同学呢?\par
    请同学分享一下自己学习过程中遇到困难的时候。或许是当他们开始主动学习一件乐器的时候或者开始接触一门语言的时候。(这个可以链接到在学习入口处提到的每一个同学的技能)。学习一些新的东西是很难并且很具有挑战的,但是如果我们肯付出,那一定会变的更好。\par
    老师也可以分享Carol Dweck的书籍《你如何充分发挥自己的潜力》。这本书有一个章节’冠军心态‘讲了在体育领域的著名人物的一些精彩的采访和观点。这本书的受众是成人,但是孩子们也可以分享其中关于体育人对待学习的态度以及努力。\par
    

\section{记录活动}
     请同学思考一下学习中遇到的挑战。可能是学习一种新的技能或者是提高一项已经会的技能(比如改善自己的写作能力)。然后接着想一下这个过程中最具挑战性的事情是什么?然后展示在’挑战专栏‘上以便全班同学都可以看到彼此学习过程中遇到的挑战。老师也需要写上自己学习中遇到的挑战,这样孩子们也能看到一个渴望挑战和进步的老师,并从中学习。\par
     每位同学都应该把自己的学习过程总结成日志。可以使用像\href{file:///D:/storage/members/temp/(www.prezi. com}{prezi.com}的展示软件来做成多媒体日志。或者使用图片、短视频的形式。这种记录可以帮助同学更好的提高自身的技能和学习。\par
     鼓励每一位同学具有分享和支持的意识。因为通过’挑战专栏‘的展示,每一个人都可以看到别人目前正在把精力放在方面,如果自己正好在这个领域很熟悉或者很精通,那么这时候可以提供建议和自己的看法。或者同学看到有相同的兴趣,可以加入到小组中一起去进步和发展。\par
     单独找出一段固定的时间来回顾同学的进步。鼓励每一位同学分享自己的学习经历和一些励志的话。班级也可以写一写励志和鼓励的话语,贴在’挑战专栏‘的旁边来帮助孩子们坚持下来。最后回顾一些’成长心态‘以及这些鼓励的话是如何帮助你成功的。


\section{个人目标}
  

   \begin{itemize}
     \item 适应
     \item 沟通
     \item 合作 
     \item 坚韧
     \item 深思 
   \end{itemize}
