\chapter{个人目标}

\section{探究活动}
     展示一下IPC的8个个人目标。同学们应该对此很熟悉了并且通过学习合作已经能够对每一个目标作出自己的解释(见里程碑1和里程碑2脑波单元)。请同学复述或者对这几个个人目标重新进行定义,然后分别给出在自己的生活中使用到这些个人目标的例子。提问个别学生,问一下在这几个个人目标中,他们自认为最擅长的目标是哪一个,最不擅长的目标是哪一个。同学们可以对根据自己的情况对这几个个人目标进行排序,从最擅长到最不擅长。或者强度从1到10,学生们自行对自己的个人目标进行强度评级。邀请志愿者分享自己的评级。\par
     比较一下同学们不同的长处和短处(个人目标),然后问几个志愿者为什么如此评级自己的个人目标。同学思考一下在未来有哪些个人目标需要进一步的加强,或者是当离开学校之后,在工作方面有哪些希望加强的方面。\par
     如果同学之前学习过一本小说、一个当下真实的故事,或者一个人的真实性格(比如说历史人物)……那么接下来的环节需要进一步完成(详细请见语言艺术链接)。新闻故事可以从人物7中选择。\par
     同学考虑这8个个人目标在故事中是如何与人物性格或者故事中的一些特定情节发生关系的。\par
     如果孩子们只有机会研究故事的概要,那讨论一下这个新闻故事或者其中的人物性格,两人一组思考场景中涉及的人或事。根据选择的计划,可以讨论的领域应该有:\par
     \begin{itemize}
       \item 故事讲了什么?
       \item 矛盾在哪里?
       \item 动机是什么?  
       \item 我们最认同的一方是谁?是正确的一方还是错误的一方?
       \item 故事中的人作出正确的选择没?
       \item 场景/冲突能否以不同方式解决? 
       \item 故事中的任务都有哪些个人目标?有分别缺少那种目标?
       \item 我们使用什么个人目标来回应已经提出的问题
      \end{itemize} 
     同学可以将自己的想法和观点记录成为一张思维导图。全班一起讨论一下这份思维导图。鼓励辩论,尤其鼓励不同的观点和看法。在可能的情况下,将讨论回溯到个人目标。例如道德就是理解是非,但是知道你是对的总是很容易吗? 可能有别人有不同的看法? 尝试并使用所提出的问题来帮助孩子加深对个人目标的理解。\par
     
\section{记录活动}
    以小组为单位,挑战排演一个新的话剧表现在课堂上探讨过的小说、新闻故事或者历史人物。不同的是将这些故事放在新的背景下————比如学校,公司,家庭等。同学应该考虑一下在心的环境下的人物和故事情节。也可以给每一个角色做一个性格卡片,卡片主要反映出这个角色的擅长的个人目标和不擅长的个人目标。(比如有些人可能社交技能很强,但是在坚韧度不够所以遇到事情很容易放弃。有一些人可能不擅长交际,但是很容易适应环境和坚持,所以可以从不同的角度来看问题。从而解决。)这种练习可以帮助同学们创造更多的不同性格的人物形象。\par
    给每一个小组时间来表演自己的话剧。在每一场话剧演完之后,全班一起讨论刚才的话剧中表达的事情,以及不同角色如何应对这种情况。有没有可能辨别每一个角色擅长和不擅长的个人目标。如果可以的话,大家对该话剧中的每一个人物形象都有什么建议帮助完善个人目标。\par
    作为这个环节的扩展部分。同学可以使用个人目标给学校整理一份‘矛盾冲突处理手册’。来帮助学校形成一个更加积极的学习环境。最终的版本可以信件或者传单的形式发放给学校中其他的成员观看。

\setction{语言艺术链接}
    选择一本短片小说(小说或者人物传记),在单元学习开始的时候可以和同学一起读,或者同学可以作为家庭作业完成。下面是推荐的一些比较有看点和有趣的小说:\par
    \begin{itemize}
      \item 《Now is the Time for Running》Michael Williams,Tamarind,2012\footnote{前面是作者名称,后面是出版社名称}
      \item  《Naughts and Crosses》by Malorie Blackman, Simon and Schuster Books 2005
      \item  2005 Holes, by Louis Sachar, Yearling Books, 2000
      \item  War Horse, by Michael Morpurgo, Scholastic Press, 2010 
      \item  Wonder, by R J Palacio, Knopf Books, 2012
    \end{itemize}  
    同学可以在完成个人目标任务之前,首先通读一下整个文章来获得更多的信息。这样他们也可以理解整部作品中涉及到的深度和广度。然后老师在上课的时候可以专门挑一部分,专门探究这一部分的细节内容。\par
    
\section{ICT链接}
    儿童可以使用\href{http://plasq.com/products/comiclife/}{漫画生活}创建他们的戏剧故事的漫画。这是一个简单易用的程序,通过将图片和照片导入之后,自动生成一个模板。在场景中也可以加上场景相关的人物对话,通过对话框的形式。


\section{个人目标}
   \begin{itemize}
      \item 适应
      \item 沟通
      \item 探究
      \item 道德
      \item 坚韧
      \item 尊重
      \item 深思 
   \end{itemize}  
