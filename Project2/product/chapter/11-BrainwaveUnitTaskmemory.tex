\chapter{记忆力}


\section{探究活动}
    同学回顾一下上一个任务关于大脑我们所学习的东西。作为探究内容的一部分,同学或许也学习了关于记忆相关的内容并且了解了关于大脑如何处理信息的过程。如果大家都了解了这一切活动的机制,请同学来分享他们的观点。如果不是,让同学们分享一些自身了解到的和记忆相关的知识。\par
    给孩子们一个挑战。准备一些比较常见的首字母缩略词(也许是文本缩写,组织,计算机术语等)比如:\par
    \begin{itemize}
      \item DNA 
      \item LOL 
      \item BBC 
      \item RAM 
      \item CNN
      \item FBI 
      \item DOB
      \item FYI
      \item SMS        
    \end{itemize}  
    将这些以单行的形式列出来。比如:\par
    DNALOLBBCRAMCNNFBIDOPFYISMS\par
    让同学20秒钟的时间去尝试记忆尽可能多的字母。然后将列表改成下面的形式:\par
    DNA, LOL, BBC, RAM, CNN, FBI, DOB, FYI, SMS\par
    然后给20秒钟时间再去记忆。问他们有没有发现现在记起来是不是比刚才要容易很多?这是为什么呢?最好是有同学虐能够知道这些缩略词背后的意思。\par
    展示下面的图片,让同学更加视觉化的体会记忆的工作机制。\par
    长期记忆是存放我们记住的一些事实和信息的地方。短期记忆实际上是一个大脑进行思考和逻辑推理的地方。我们的短期记忆空间极度有限经常会被用尽所占的所有空间。所以当我们去记忆第一个形式的字母的时候实际上就是将所有的字母一下子塞进大脑的短期记忆区。很不现实。\par
    相反,如果我们将这些单词进行拆分,并对其赋予意思。比如‘LOL’可以看成是‘Lots of Love’\footnote{大声笑}。这个时候,我们实际上是从我们的长期记忆中提取出一些已有的知识来帮助记忆,缩短了短期记忆的长度(不是27个单词而是9个单词)。\par
    这种将信息一块块系在一起的做法叫做“chunking”\footnote{分块}。这是一种将大脑中的知识变的更好消化的方式。这也是为什么如果我们对一个主题很熟悉,我们便会很快速的很容易的思考与其相关的很多内容的原因。主要是因为我们可以很快的将所需的信息从长期记忆中提取出来,也意味着我们的记忆空间会变充足。\par
    你也可以尝试另一种游戏,告诉同学即将的游戏是每人将需要用日语数到五。然后老师将使用日语非常快速的从一数到五。然后问一下同学是不是有人能够重复一遍老师。然后,给给每一个日语数字配上相应的动作,让学生尝试模仿这些动作。比如:\par
    \begin{itemize}
      \item ichi(一)——听起来像'itchy'\footnote{英文:瘙痒的意思,这里指的是发音一样},所以做一个挠头的动作。
      \item You(二)——听起来像‘knee’\footnote{英文:膝盖,这里指的是发音一样},所以指一下你的膝盖。
      \item San(三)——听起来像‘sun’\footnote{英文:太阳;这里指的是发音一样},所以指一下天上。
      \item You(四)——听起来像‘yawn’\footnote{英文:打哈欠;这里指的是发音一样},所以可以做一个打哈欠的动作。
      \item Go(五)——听起来像‘Go’\footnote{英文:跑;这里指的是发音一样},所以可以做一个让别人走开的动作。
    \end{itemize}  
    重复这些动作,然后让孩子们独立完成这些动作。他们现在能够从1数到5吗?这些动作是如何帮助他们记住这些数字的呢?这次让同学自己解释一下长期记忆(那些在脑子里固话的事实和记忆)是如何帮助短期记忆处理信息的。\par
    然后向同学解释道:这也是为什么使用比喻句‘像……’这种话更能促进学习。这样能将以前的记忆和现在学习的新东西联系起来。\par
    给出一两个动作比如‘高峰就好像很多很大的小山丘’,‘篮球就像netball\footnote{类似篮球的运动}’,'有轨电车就像在路上跑的火车'。让每一位同学都想’x好像是y‘这样的例子来帮助人们理解不能够理解的东西。如果班上有不同国家的学生,老师可以请这些同学来用别的的文化解释一下自己国家的文化。\par
    小组为单位,或者班级为单位。探索出更多的促进记忆的方式。\par
    以下的网址和链接会提供帮助:\par
    \begin{itemize}
       \item \href{http://www.bbc.co.uk/scotland/brainsmart/memory/}{bbc.com}- BBC Brain Smart网站提供了有关如何使事实更容易学习的有用视频。 还有其他文章,其中包含不同记忆技术的示例。 你知道你的石笋来自你的钟乳石吗?
       \item \href{http://www.bbc.co.uk/blogs/legacy/scotlandlearning/2010/04/make-themost-of-your-memory.shtml}{bbc.com} BBC学习网站主持了几个视频,内容专家Drew McAdam分享了记忆信息的不同技巧 - 包括人名和洗牌。
       \item \href{http://www.thememorypage.net/}{thememorypage.net}记忆页面提供记忆提示和一系列记忆游戏来测试你的能力。
       \item \href{http://www.thelearningweb.net/memory-tricks.html}{thelearningweb.net}学习网提供了一个简单的流行记忆技术清单。
    \end{itemize}  
    鼓励孩子思考这些技巧如何能够在课堂上使用以便来帮助自己的学习。比如:应对即将到来的考试。\par

\section{记录活动}
    班级为单位或者个人为单位,在规定的时间内完成一些挑战。每位同学都选择一种帮助自己完成挑战的记忆技巧。需要确保这不是一次竞争,而是一次娱乐活动或者你自己探究出的技巧的一次实验。记忆只不过是一种技巧而已,这需要被练习——但完全没有必要成为这方面的专家。\par
    挑战包括:\par
    \begin{itemize}
      \item 记忆多张扑克牌
      \item 记忆一列名字或者一个事项安排表(比如婚礼)
      \item 记忆一些书籍中传播的事实(比如儿童百科全书)
      \item 记忆国家的首都 
    \end{itemize}  
    老师可以在孩子们完成任务的过程中进行拍摄,采访他们来寻找到更加实用的记忆技巧。然后在结尾,给每一个小组规定的时间来展示记住了多少。再次申明————确保这次活动的娱乐性。有一些同学或许发现这些任务是很难的,而且很有压力,这样的话他们很难坚持下来。这时候,老师们应该鼓励孩子,并给他们加油打气,让他们有信心将挑战展示下来并谈论一下自己使用的技巧。\par
    其中有人能够完成挑战吗?或者谁最接近成功?一起探讨一下使用的技巧和思考这些技巧能不能在课堂上推广使用。\par

\section{个人目标}
   \begin{itemize}
     \item 适应
     \item 沟通
     \item 探究 
     \item 深思  
   \end{itemize}  
