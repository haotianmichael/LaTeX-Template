\chapter{国际化任务-1}

\section{学习目标}
\begin{itemize}
  \item 对于同学来讲,了解他们各自的祖国之间有什么不同和相似之处。以及了解自己的祖国和居住的国家之间有什么不同。
  \item 了解这些不同和相似点是如何影响人们的生活的。
  \item 能够识别与自己祖国不同但相等的活动和文化 
\end{itemize}



\section{探究活动}
   许多可可豆的农民因为得到的报酬实在是太少了,所以就让自己的孩子去工作,从而养活家里人。回忆一下地理章节扩展部分。\par
   应该怎么做才能确保这些农场主得到合理的收入。这算是一个国际化的问题吗————一个需要消费巧克力的国家共同解决的问题?\par
   帮助同学们找到关于公平贸易协会,这些组织一直致力于为可可豆的种植者获取合理的收入。\par
   查看东道国或本国主要巧克力制造商的网站,了解他们是否购买公平贸易可可豆。 看看公平贸易标志的巧克力包装纸。 如果无法找到所需信息,同学们写信给制造商,了解他们是否有公平交易购买政策。\par
   下面的网址会帮助学生开始一个新的话题:\par
   \begin{itemize}
     \item \href{https://www.nestle.co.uk/csv2013/socialimpact/responsiblesourcing/nestlecocoaplan}{nestle.com}雀巢网站提供有关其可持续可可计划的信息。
     \item \href{http://www.dagobachocolate.com/circle.asp}{dagobachocolate.com}DAGOBA有机巧克力网站提供有关其公平贸易巧克力的信息。
     \item \href{http://www.thehersheycompany.com/social-responsibility.aspx}{thehersheycompany.com}Hershey的网站上有关于其社会责任的信息。
     \item \href{https://www.cargill.com/corporate-responsibility/pov/cocoa-sourcing/index.jsp}{www.cargill.com}- 嘉吉网站提供有关其公司可持续采购可可豆的信息和视频。
     \item \href{http://www.papapaa.org}{papappaa.com}Pa Pa Paa网站提供丰富的免费,全面的课程计划和资源,用于教授公平贸易和巧克力。
   \end{itemize}


\section{记录活动}
    我们是否足以帮助可可豆的种植者。请孩子们将自己的研究成果做成一份报告。\par
    从这次探究学习中,同学都学习到了什么?他们的家庭,学校,他们的IPC社区和他们的国家还能做些什么?向全班收集建议和想法观点。\par
    技术、艺术链接:在制作巧克力的时候,使用的原材料是不是符合公平贸易法则?如果没有,这一点可以作为真个活动可以改善的地方,结束之后可以退出学习出口。

\section{个人目标}

\begin{itemize}
  \item 沟通
  \item 探究
  \item 道德
  \item 尊重
  \item 深思
\end{itemize}
