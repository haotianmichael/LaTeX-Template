\chapter{科学任务1}

\section{学习目标}

\begin{itemize}
  \item 能够通过简单的文本收集信息。
  \item 了解到营养,生长,运动和繁殖的本质。
  \item 了解坚持锻炼对人体的重要性。
\end{itemize}  


\section{探究活动}
   向孩子们展示一块巧克力。问一下他们吃巧克力对身体是好还是不好?当然了,回答当然不一样(说有好处因为巧克力提供能量,没有好处是因为吃巧克力会变胖)\par
   巧克力的原材料是什么?回顾一下地理单元,然后看一下巧克力包装袋上的营养成分。下面是一个样品:
   \begin{table}[h]
     \begin{tabular}{l|l}
     \hline
     每100克的典型值
     能量 & 525千卡路里
     蛋白质 & 5.4克
     碳水化合物 &  60克
     蔗糖 & 59克
     油脂 & 29克
     纤维 &  2.2克
     钠 & 0.06克
     \hline
   \end{tabular}
   \end{table}

   巧克力中的主要的原料是蔗糖和油脂。同学们可以两个人一组探究一下身体是如何运用油脂和蔗糖的。人体每天需要多少的蔗糖和油脂?下面的网址可能会帮到你:\par
   \begin{itemize}
      \item \href{http://www.bbc.co.uk/northernireland/schools/4_11/uptoyou/healthy/nutrientfacts5.shtml}{bbc.co.uk}BBC网站提供有关脂肪和糖的信息,并与社区营养师进行播客采访。
      \item \href{kidshealth.org/kid/stay_healthy/index.html#cat119 }{kidshealth.com}KidsHealth网站有一系列关于健康饮食的功能。
      \item \href{http://www.choosemyplate.gov}{choosemyplate.com}My Plate网站有关于主要食物组的信息。
   \end{itemize}

   一块巧克力提供了很高的能量。如果你的身体立刻释放掉这些能量,你会像炸药一样爆炸。但是不要担心这是不可能的,因为你的身体将会很慢的消耗这些能量,身体所做的一切事情都是让你的身体很好的很健康的运转。\par
   你每天需要的能量取决于很多因素,包括你每天的运动量。你的身体每天都会用到能量,就算你在睡觉的时候都会消耗能量。和班上的同学讨论一下有没有很快的消耗你的身体的能量的行为,比如踢足球,健身,跳舞,跑步等。\par
   世界健康组织建议5-17岁的孩子每天需要坚持做60分钟的剧烈运动来保持身体的健康。(这是一个积累的过程)讨论一下什么是剧烈运动?比如:跑步,跳绳,踢足球等等。\par
   

\section{记录活动}
     将这些东西展示在班上的白板上或者知识获取部分。\par
     \begin{itemize}
       \item 吃掉一块巧克力的时间:一分钟
       \item 消耗掉来自巧克力的能量:
       \item 跑步:14分钟
       \item 走路:52分钟
       \item 游泳:24分钟
       \item 看电视:4小时
       \item 睡觉:5小时
     \end{itemize}  
     同学们可以将这些资料记下来,做成表格和图。在研究问题的时候可以参考。然后,两个人一组,回答一下面的问题:\par
     \begin{itemize}
       \item 如果你打算跑马拉松,那在跑步之前或者跑步之后吃一块巧克力是不是一件好事?
       \item 问什么通过运动来平衡吃巧克力的热量这么重要?
       \item 那些身体不需要消耗的热量都发生了什么?
       \item 在睡觉前吃巧克力好不好?
       \item 如何将巧克力作为营养平衡餐中的一部分?
     \end{itemize} 

     科学/语言艺术链接:让同学通过书籍和网络查找一些关于消化系统的信息,然后了解一下我们的身体到底发生了什么?基于这些调查,孩子们可以制作一个带有注释的作品————一块巧克力在人身体的旅行,或者还可以制作一话剧来阐述发生了什么?下面书籍可能是一个比较好的开始:\par
     \begin{itemize}
       \item See Inside Your Body (Usborne Flap Books), by Katie Daynes, Usborne Publishing, 2006
     \end{itemize}  
     


\section{个人目标}
   \begin{itemize}
     \item 沟通
     \item 探究
     \item 尊重
     \item 深思
   \end{itemize}  




