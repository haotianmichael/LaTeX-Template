\chapter{技术任务-2}



\section{学习目标}
  \begin{itemize}
    \item 知道日常使用中的产品的设计方式是如何影响他们的用途的
    \item 能够按照需求设计和制作产品
    \item 能够制作有用的计划
    \item 能够精确的使用简单的工具和设备
    \item 能够识别并实施其设计和产品的改进
    \item 能够识别日常使用的产品满足一些特定需求的设计方式
    \item 能够对日常用品的设计提出一些建议
  \end{itemize}  
   


\section{探究活动}
    我们经常在超市中买到的巧克力一般都可以加什么成分?\par
    将孩子们分成小组,邀请他们来思考还可以向巧克力中继续加的成分。\par
    给他们提供一些水果干和花生(小心过敏)。同学可以自行完成这些物质的混合,然后将邀请老师和其他同学来作为品尝师来品尝这些,然后评价出最受欢迎的搭配。卫生教育要时时刻刻保持和遵守。\par
    老师可以建议同学使用向日葵籽和切碎的杏或者葡萄干和花生。\par
    

\section{记录活动}
  同学们应该将最受欢迎的搭配记录下来,然后为这个新产品做一个产品的推广计划书。\par
  \begin{itemize}
     \item 这种巧克力会吸引什么人?
     \item 谁会买?
     \item 产品的包装袋可以使用什么材料?
     \item 包装袋应该如何设计?
     \item 制作巧克力的成本会是多少?
  \end{itemize}  
  在选择最好的包装袋材料的时候,同学应该参考科学任务-5。在设计一个比较好看的包装袋的时候,同学应该参考艺术任务-1和-2。\par
  

\section{个人目标}
\begin{itemize}
  \item 适应
  \item 沟通
  \item 探究
  \item 深思
\end{itemize}
