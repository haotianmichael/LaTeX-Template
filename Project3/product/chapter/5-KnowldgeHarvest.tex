\chapter{知识获取}
    提前准备一些威利卡旺巧克力的包装纸,要班上每一位同学人手一份。然后要求每一位同学自己使用计算机设计软件自己设计并打印出来一份。其中隐藏一些金色包装纸(装饰着金色星星)。在这些金色的巧克力包装纸背后写上一些关于巧克力的事实————比如巧克力(可可)豆在豆荚内生长等。其他的包装纸的背后什么也不要写。\par
    将这些包装纸装在一个帽子里(威利卡旺的帽子)然后邀请孩子们闭上眼睛自行选择。如果他们拿到了那些金色包装纸,便在班上大声朗读这些包装纸背面写的东西。如果他们拿到的是那些什么也没有写的普通包装纸,那他们需要在巧克力包装纸背面自己写上一些关于巧克力的事实并在班上大声朗读出来。每一位同学不能重复已经朗读过的事实,所以客观上讲这个游戏会越进行越困难。 在游戏结束后,孩子们需要将刚刚朗读过的事实作为材料展示在本单元的知识获取一栏中。\par
    观看查理和巧克力工厂DVD的摘录。选择巧克力制作的过程观看。然后,问孩子们想象如果自己正在参观一家巧克力工厂——不是威利卡旺的工厂。
    \begin{itemize}
        \item 他们希望看到什么?
        \item 都是什么会来到工厂?
        \item 最后会变成什么?
        \item 工厂的原材料是什么,最后做出的又是什么?
        \item 工厂最终的产品是什么?
        \item 产品是如何包装和打包的?
    \end{itemize}  
    邀请班上的一些孩子们画出一个巧克力制作机器。\par
    从这些绘画作品和写在包装纸背后的事实中,老师可以将孩子们的观点做成一系列表格。然后展示在班级上以便于在以后的学习过程中不断的总结和回顾。随着单元学习的不断推进,孩子们关于巧克力制作生产过程的认知也会不断深入。这时候他们可以适时地的修改之前关于巧克力的猜想。\par
    
