\chapter{知识图谱}
   


\section{可可}
     可可发源于玛雅文明,主要生长在危地马拉。是一种价值很高的商品。在加工之后,可可树上的豆子会被制作为巧克力饮品中的巧克力。想我们一样,玛雅人也喜欢巧克力,不过不同的是我们将巧克力做成巧克力棒来吃,而玛雅人是喝!他们用辣椒等香料调味巧克力饮料,有时他们会用蜂蜜。他们确保自己的音频会有很多的泡沫,我们在很多地方看到这种描述。我们也看到过一些关于玛雅人如何制作者写又泡沫的饮品————将液体从高处倒进另一个容器中。可可豆在烘烤之后,便于存储和运输。也正在因为这个原因,可可在后经济学和接触时期的巨大市场经济中成为一种很重要的交换媒介(货币)。从这个角度来讲,玛雅人实际上是在喝钱!\par
     将可可作为一种货币和饮品的传统被由中美洲和南美洲的阿兹特克人延续下来。这帮人将这种饮品成为“chocolatl”\footnote{巧克力的英文是chocolate}。该饮品被作为“神的食物”并放在金被子中作为皇家贡品。据说皇帝蒙特祖玛一天要喝50杯!可可不仅是一种很招人喜爱的饮品。更是玛雅人和阿兹特克人宗教活动和社会活动中很重要的组成部分。\par

\section{哥伦布和科尔特斯}
     16世纪的探险家哥伦布,在1504年的时候,从当时的新世界————就是现在的美洲带了一些可可回到西班牙。西班牙国王和皇后并没有认识到哥伦布带回来的这些豆子的重要性,他们倒是很看重哥伦布带回来的黄金和奇珍异宝。豆子的重要性被1519年的探险家科尔斯特发现了。刚开始蒙特祖玛国王赠予很多的可可豆,但科尔斯特并不喜欢。直到他发现这些豆子可以被作为当地的货币使用的时候(4个豆子换一只兔子,10个豆子换一个奴隶),这时候他才意识到这一发现的重要性————他可以在树上种出钱来。

\section{关于埃尔南 科尔斯特的更多信息}
     科尔斯特是一位西班牙的探险家,他在1521年的时候征服了阿兹克特王朝。科尔斯特利用自己洁白的皮肤和小胡子让蒙特祖玛二世国王相信自己就是神。上古的预言成真了。而且国王也相信了这件事情。但是之后科尔斯特并没有表现的像一个神一样。他们被驱逐出境。但是科尔斯特不久之后又回来了,这次他带了600名士兵。他捉住了蒙特祖玛二世国王并毁掉了阿兹特克王朝的首都特诺奇蒂特兰城。并在废墟上建立了墨西哥城。原始城市的所有遗迹都是主殿的废墟。


\section{巧克力酱}
     从韦拉克鲁斯到塞维利亚的巧克力官方装箱发生在1585年。在17世纪的欧洲贵族中,辣椒被替换成为糖从而变得更加流行了。英国人,法国人和德国人将这些可可引进到自己的殖民地。在美国,巧克力是在马萨诸塞州多切斯特首次制作的。第一个巧克力棒是由弗莱在1847年在伦敦制作的。两年后,吉百利制作了类似的产品。 牛奶巧克力是由Daniel Peter于1876年在瑞士制造的,只需加入奶粉即可。第一次世界大战后几年,雀巢推出了白巧克力。它含有可可脂,但根本没有可可固体。 出于这个原因,有些人认为它不应该被称为“巧克力”。今天,世界上80%的巧克力仅由六家跨国公司生产,包括雀巢,火星和吉百利。最大的巧克力消费者是瑞士,每人每年消费超过10公斤(22磅)。 排名前十位的消费者包括:瑞士,德国,奥地利,爱尔兰,英国,挪威,爱沙尼亚,斯洛伐克,瑞典和哈萨克斯坦。


\section{主打产品}
      




\section{热带生物群系}



\section{可可树}


\section{收获}



\section{}
