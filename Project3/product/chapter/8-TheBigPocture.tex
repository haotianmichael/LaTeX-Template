\chapter{知识图谱}
   


\section{可可}
     可可发源于玛雅文明,主要生长在危地马拉。是一种价值很高的商品。在加工之后,可可树上的豆子会被制作为巧克力饮品中的巧克力。想我们一样,玛雅人也喜欢巧克力,不过不同的是我们将巧克力做成巧克力棒来吃,而玛雅人是喝!他们用辣椒等香料调味巧克力饮料,有时他们会用蜂蜜。他们确保自己的音频会有很多的泡沫,我们在很多地方看到这种描述。我们也看到过一些关于玛雅人如何制作者写又泡沫的饮品————将液体从高处倒进另一个容器中。可可豆在烘烤之后,便于存储和运输。也正在因为这个原因,可可在后经济学和接触时期的巨大市场经济中成为一种很重要的交换媒介(货币)。从这个角度来讲,玛雅人实际上是在喝钱!\par
     将可可作为一种货币和饮品的传统被由中美洲和南美洲的阿兹特克人延续下来。这帮人将这种饮品成为“chocolatl”\footnote{巧克力的英文是chocolate}。该饮品被作为“神的食物”并放在金被子中作为皇家贡品。据说皇帝蒙特祖玛一天要喝50杯!可可不仅是一种很招人喜爱的饮品。更是玛雅人和阿兹特克人宗教活动和社会活动中很重要的组成部分。\par

\section{哥伦布和科尔特斯}
     16世纪的探险家哥伦布,在1504年的时候,从当时的新世界————就是现在的美洲带了一些可可回到西班牙。西班牙国王和皇后并没有认识到哥伦布带回来的这些豆子的重要性,他们倒是很看重哥伦布带回来的黄金和奇珍异宝。豆子的重要性被1519年的探险家科尔斯特发现了。刚开始蒙特祖玛国王赠予很多的可可豆,但科尔斯特并不喜欢。直到他发现这些豆子可以被作为当地的货币使用的时候(4个豆子换一只兔子,10个豆子换一个奴隶),这时候他才意识到这一发现的重要性————他可以在树上种出钱来。

\section{关于埃尔南 科尔斯特的更多信息}
     科尔斯特是一位西班牙的探险家,他在1521年的时候征服了阿兹克特王朝。科尔斯特利用自己洁白的皮肤和小胡子让蒙特祖玛二世国王相信自己就是神。上古的预言成真了。而且国王也相信了这件事情。但是之后科尔斯特并没有表现的像一个神一样。他们被驱逐出境。但是科尔斯特不久之后又回来了,这次他带了600名士兵。他捉住了蒙特祖玛二世国王并毁掉了阿兹特克王朝的首都特诺奇蒂特兰城。并在废墟上建立了墨西哥城。原始城市的所有遗迹都是主殿的废墟。


\section{巧克力酱}
     从韦拉克鲁斯到塞维利亚的巧克力官方装箱发生在1585年。在17世纪的欧洲贵族中,辣椒被替换成为糖从而变得更加流行了。英国人,法国人和德国人将这些可可引进到自己的殖民地。在美国,巧克力是在马萨诸塞州多切斯特首次制作的。第一个巧克力棒是由弗莱在1847年在伦敦制作的。两年后,吉百利制作了类似的产品。 牛奶巧克力是由Daniel Peter于1876年在瑞士制造的,只需加入奶粉即可。第一次世界大战后几年,雀巢推出了白巧克力。它含有可可脂,但根本没有可可固体。 出于这个原因,有些人认为它不应该被称为“巧克力”。今天,世界上80%的巧克力仅由六家跨国公司生产,包括雀巢,火星和吉百利。最大的巧克力消费者是瑞士,每人每年消费超过10公斤(22磅)。 排名前十位的消费者包括:瑞士,德国,奥地利,爱尔兰,英国,挪威,爱沙尼亚,斯洛伐克,瑞典和哈萨克斯坦。


\section{主要的生产者}
     可可树生长在西非。美洲中部南部以及东南亚地区。主要的生产者有:象牙海岸,加纳,印度尼西亚,尼日利亚,喀麦隆,巴西,厄瓜多尔和马来西亚。

\section{热带生物群系}
     可可树生长在赤道附近的热带地区,所以季节性变化不是很明显。这边的气候一直都热,很少有平均月份低于25度的月份。而且降雨量空气湿度都很高。这里的土壤因为多年的动物和植物死亡之后的尸体沉淀物的积累,一直很富饶。所有这些自然环境,都是可可树和其他一些植物生长的绝好温室。

\section{可可树}
     可可树的生长非常脆弱。生长需要温度、水源还要防止大风和强烈的阳光等。这种植物在热带雨林的大树下面生长是最好的,好的话可以生长到6到10米高。当长到4到5年的时候,可可树会开出粉红色的花。一种通常被称为“豆荚”的水果(长约15厘米,形状像橄榄球)在短茎上靠近树生长。每个水果上面30-40个种子(可可豆)。这种水果刚开始还是绿色的,当变得逐渐成熟的时候,颜色就会变黄。可可豆是紫色的,里面的果肉是白色的。在野外,猴子们会去吃果肉,从而帮助种子传播。

\section{收获}
      首先用小刀收集水果,然后带到镇上去。再切的时候,一定要小心豆不要被切碎了。最后豆子去掉果肉,放在香蕉树叶下面催熟。一周后,这些豆子变成棕色的,这个时候看起来,闻起来都像可可的味道了。放到太阳下面去晒,等到豆子晒干了,就可以放到一个亚麻袋子里拿去工厂加工了。

\section{加工}
     可可豆最后被制作成为巧克力。最开始要先清洁,清洁后拿进一个大的炉子里烘烤。最后这些豆子会被放进一个大的机器中,被做成为浆糊状。将这些浆糊挤压,挤出的油叫做可可黄油。将黄油,白糖和浆糊放进一个盘子里搅拌超过大概2-3天。然后原装巧克力就做成啦!要做成20条巧克力,你需要整整一棵可可树。


\section{经济作物}
    在一个国家,经济作物种来不是为了吃的,而是为了卖到另一个国家的。经济作物的种植和培养对自然环境有巨大的影响。在过去的50年内,一些发展中国家已经开始大面积种植单一的作物比如:棉花,玉米,大豆,甜菜和可可。有时候这些土地并不适合一些作物的生长。或者有些土地需要被灌溉,或者有些土地需要控制虫害。这时候,大公司往往会收购这些土地,因为他们有能力来提供这些技术和资金。但是有一些没有经济能力的贫穷农民,会因此去到森林中,大量的砍伐树木来获得更合适的耕地。所以很多的森林都变成了经济作物的耕地,所以一些贫穷的国家就没有能力去承担自己国家的人口增长带来的健康和营养不良等问题。


\section{公平交易}
    一些经济作物的农场主并不能得到公平的报酬。往往回报很少。公平交易相关的组织和政策旨在帮助这些农民获得自己种植的经济作物的合理价格。


\section{热带雨林}
    全世界每年大约有5千万英亩的热带雨林在消失,大约是一个埃及的面积。按照现在这样的一个速度,在2050年,全世界将会只有5\%的热带雨林会留下来。
    热带雨林栖息地很重要是因为:\par
    \begin{itemize}
       \item 热带雨林是地球上存在最久的生态系统。他们以这种形式存在已经超过7千万到一亿年了之久了。
       \item 热带雨林是地球上5百万到一千万物种当中超过一半的物种的家园。
       \item 在世界热带雨林中,平均每天有137种生命正在灭绝。
       \item 今天可用的药物中可能有一半要归功于热带雨林中的有的原材料植物。
       \item 热带雨林是地球上气候的控制者,如果热带雨林消失了,那水循环就会被打破,干旱就会到来,沙漠会形成。
       \item 热带雨林被称作是“地球之肺”,是因为据统计热带雨林供给了地球上20\%的氧气,最近一些科学家反对这个结论,但是他们都一致同意热带雨林的消失会给地球的未来带来可怕的后果。
    \end{itemize}  
    

    

