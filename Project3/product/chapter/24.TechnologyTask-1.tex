\chapter{技术任务-1}


\section{学习目标}
 \begin{itemize}
    \item 知道日常使用中的产品的设计方式是如何影响他们的用途的
    \item 能够按照需求设计和制作产品
    \item 能够制作有用的计划
    \item 能够精确的使用简单的工具和设备
    \item 能够识别并实施其设计和产品的改进
    \item 能够识别日常使用的产品满足一些特定需求的设计方式
    \item 能够对日常用品的设计提出一些建议
  \end{itemize}  
   


\section{探究活动}
    如果我们需要自己制作巧克力,我们会需要什么材料?想全本同学征求一下建议。如果需要探究巧克力包装袋上的成分的话,回顾一下科学任务1。\par
    告诉孩子们这单元他们将会自己制作巧克力。给他们制作的食谱和流程。老师需要:\par
    \begin{itemize}
      \item 两汤匙粉状可可
      \item 两汤匙蔗糖—- 水果糖,适合光滑的巧克力
      \item 一茶匙无盐黄油或蔬菜起酥油
      \item 一个双层蒸锅或双锅炉
      \item 蜡纸
    \end{itemize}
    \par步骤\par
    \begin{itemize}
      \item 在大人的帮助下,将沸水倒进碗里
      \item 关掉热水器,将各个成分倒进水表面
      \item 搅拌直至混合物变得光滑 
      \item 将混合物倒入蜡纸上
      \item 让它硬化,切割和品尝 
    \end{itemize}
    链接到科学扩展部分,当你观察到融化的巧克力慢慢的变成固态的时候。\par
    同学可以尝试改变每种成分的含量来改变巧克力的味道。比如多加可可豆就可以得到黑巧克力,请问同学们如何制作白巧克力?(使用加牛奶)。



\section{记录活动}
    孩子们制作巧克力的过程可以拍摄成为视频或者照片的形式。同学需要给这些视频加上字幕和标题或者一步一步的步骤指令,通过这种形式来向班上的成员和家长解释巧克力的制作方式和流程。\par
    有一个“品尝会”,问同学们如果有下一次机会来制作自己的巧克力,他们会不会有什么新的想法和提高的技巧来补充改善质地或风味。\par
    高年级的学生可以挑战一下试用模具来改变巧克力的固定形状,作为扩展部分,可以请教这些同学在这方面的看法和观点。\par
    下面的网址解释了如何使用气球来制作巧克力杯的形状。\par
    \begin{itemize}
      \item \href{http://www.exploratorium.edu/exploring/exploring_chocolate/activity.html}{exploratorium.com}Exploratorium使用巧克力探索不同的活动。
    \end{itemize}  



\section{个人目标}
  \begin{itemize}
    \item 沟通
    \item 探究
    \item 深思  
  \end{itemize}  
