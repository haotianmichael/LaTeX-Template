\chapter{历史扩展任务}

\section{学习目标}
  \begin{itemize}
      \item 通过研究了解到一些关于古时代的主要大事件、纪时和历史人物。
      \item 了解在某个特定时期人们的生活。
      \item 通过研究了解不同时代的人类生活的相似点和不同点。
      \item 能够给出一些特定事件的具体原因和转折点及其原因。
      \item 能够从简单的来源收集信息。      
      \item 能够使用自己所学的知识和理解对过去做出合理的解释。
  \end{itemize}

\section{扩展任务}
   最开始,西班牙人想将巧克力作为自己国家独有的一种秘密饮料。但是过了没多久,欧洲其他的国家和探险者们也都听说了这些事情。截止到17世纪为之,巧克力是欧洲皇室和贵族当中最受欢迎的饮品。英国,法国,西班牙和德国都开始根据自己国家的需求进行了可可豆的种植。但是这项工作确实需要很多的细节。所以这些殖民者就将非洲人作为奴隶贩卖来帮助自己种植和培养可可豆农场——这些奴隶也做咖啡豆、棉花和蔗糖。\par
   和同学一起沿着“三角贸易”的路线回忆一遍。这或许需要用到一些地理单元的知识和一些经济作物的历史。\par
   下面的网址提供了一个很好的开始。\par
   \begin{itemize}
      \item \href{http://www.nmm.ac.uk/freedom/viewTheme.cfm/theme/triangular}{nmm.ac.uk}国家海事博物馆有一个互动的“三角贸易”地图。
      \item \href{https://en.wikipedia.org/wiki/Atlantic_slave_trade}{en.wikipedia}维基百科提供有关奴隶贸易的照片和信息。
      \item  \href{https://africa.mrdonn.org/slavetrade.html}{afriac.com}Donn先生网站上有关于大西洋奴隶贸易的特色和课程计划。
      \item  \href{http://www.bbc.co.uk/bitesize/ks3/history/industrial_era/the_slave_trade/revision/4/}{www.bbc.com}BBC Bitesize网站上有关于奴隶贸易的地图和信息。
      \item  \href{http://www.theschoolrun.com/homework-help/the-atlantic-slave-trade}{theschoolrun.com}School Run对奴隶贸易有很好的概述,附有图片和时间表。
      \item  \href{http://abolition.e2bn.org/slavery_43.html}{abolition.e2bn.org}废除项目提供有关三角奴隶贸易三个阶段的信息。
      \item  \href{http://www.liverpoolmuseums.org.uk/ism/slavery/}{www.liverpoolmuseums}国际奴隶制博物馆网站包含和奴隶船上生活相关的事情,以及奴隶的第一人称个人帐户。
   \end{itemize}
   允许同学以自己选择的方式完成研究报告。比如,可以选择创建一套话剧或者交互式的个人展示。研究一场卖座的角色扮演(以以前的学习报告为经验基础)或者一张带有注释,地图和时间线的大海报。\par
   语言艺术链接:孩子们可以完成一套简短的话剧来表现一个奴隶孩子的生活。如果他们知道他们将被带到一个很危险的岛屿上,或许这辈子再也不会和家人相见,那他们会如何想。
   

\section{个人目标}
  \begin{itemize}
     \item 适应
     \item 沟通
     \item 探究
     \item 道德
     \item 深思
  \end{itemize}  
