\chapter{地理任务-1}



\section{学习目标}
    \begin{itemize}
      \item 了解居住国家的天气和气候条件,以及这些条件是如何影响当地环境和当地人们的生活的。
      \item 能够使用地理专业术语
      \item 能够使用不同层面的地图来对一些特定的地理位置进行定位
      \item 能够通过第二信息员获得地理信息
      \item 能够谈论一些关于地理方面的知识,和 回答一些关于自然条件和地理特征方面的问题
      \item 理解小地区是如何适应比较大的地理版图的
    \end{itemize}  


\section{探究活动}
    巧克力是怎么做出来的?\par
    有些同学会觉得巧克力是在树上长出来的(回忆一下知识获取单元)。在白板上画出一棵树来,在树的枝头长满了巧克力果实。然后问一下班上的同学,和他们想的一样吗?然后向同学们显示真正的可可树,大家会发现,通过在Google图片上输入“可可树”这样的关键词会出现很多类似的图片。\href{http://images.google.com/}{google.com}  \par
    同学还可以接着搜索下面的这些内容:\par
    \begin{itemize}
      \item 巧克力(可可)树生长在哪里?
      \item 我们可以在学校的空地上或者当地的空地上种这些可可树吗?我们可以在自己的祖国种可可树吗?
      \item 可可树的生长需要哪些气候/土壤条件?
    \end{itemize}
    
    下面的书籍和网址为进一步探究提供一些有用的资源:\par
    \begin{itemize}
      \item A Chocolate Bar, How It’s Made series, by Sarah Ridley, Franklin Watts, 2009 
      \item Smart about Chocolate, by Sandra Markle, Grosset & Dunlap, 2004 
      \item Chocolate: Riches from the Rainforest, by Robert Burleigh, 2002
      \item \href{http://www.cacaoweb.net/cacao-tree.html}{可可网站有一个关于巧克力生产和制造的照片。}
      \item \href{http://www.candyusa.com/story-of-chocolate/}{“巧克力的故事”网站有一个“来源”部分,其中包含有关巧克力起源的信息,照片和视频。} 
      \item \href{http://www.wildernessclassroom.com/students/archives/2006/03/chocolate_treec.html}{荒野教室有关于可可树的图像和事实}   
    \end{itemize}  
    同学会发现可可树只能生长在南美,中美洲,西非,东南亚和其他一些气候条件比较湿热的热带地区(有固定的降雨 ),这些地区土壤条件一般来说都会比较肥沃。老师通过这个机会,让孩子们了解一下植物的生长都需要哪些条件(空气,光照,水源,来自土壤的营养和适合生存的空间)。通过这些对可可树的研究,同学们应该明白,对于不同的植物,不同的树木,它们生长起来的条件都是不一样的。\par
    找一张世界气候地图,定位一下自己的国家和居住的国家,然后在找一些可可的生产地比如多明尼加共和国。\par
    思考可可树的生产地一般需要什么样的条件,在世界气象协会网站上搜索天气报告:\par
    \begin{itemize}
      \item \href{http://worldweather.wmo.int/index.htm}{世界气象协会有地图,月平均气温和总降雨量数据。}
    \end{itemize}  
    真正的热带气候又是什么样的?向孩子们讲述一下什么是温室?如果有机会可以亲自去参观一下————一些当地的公园和花园都有温室。

\section{记录活动}
    首先,让同学在地图上找到赤道。然后问一下他们可可树的主产地在在大纲世界地图上都是什么颜色————(象牙海岸,加纳,尼日利亚,巴西,厄瓜多尔,委内瑞拉,多米尼加共和国,巴布亚新几内亚,印度尼西亚)。比较一下和世界气候地图之间的不同。会有什么发现?(主产地是热带地区和赤道周边的国家)。\par
    同学通过听天气预报,在地图上给这些国家中的其中一个做出标注。然后再听一下自己居住国家的天气预报,做一下对比看有什么区别。\par
    高年级的学生或许还能够解释一下气候和天气的区别。什么是一个国家的气候,什么又是天气?(气候指的是一个国家多年以来的平均天气状态,而天气主要指的就是具体的像温度、降雨和风速等这些的度量)。\par
    最后总结的时候,问一下同学我们能不能在当地种植可可树?为什么能?为什么不能?\par
    数学链接:这个任务为那些高年级或者更有能力的同学提供了“平均测量”的概念。他们可以试着去比较可可主产国家和自己的国家的平均降雨和平均温度。画出表格来比较这两个地方的气候。计算一下差距。\par
    科学链接:如果我种了一块巧克力,那它会生长吗?如果我种了一颗可可豆,那它会生长吗?如果我种了一棵可可树,那它会生长吗?\par
    收集一下班上同学们的信息,试着画一张可可树和当地的其他树木(比如果树)的不同。\par
    两人一组,同学们来探究:\par
    \begin{itemize}
      \item 可可树长什么样?平均高度?果实?种子?叶子?开的花?等
      \item 生长的条件(温度、湿度、阴影等)
      \item 可可树是如何繁殖的?(在野外,猴子在吃果肉的时候会将种子散落在外面达到传播的效果。在种植地,可可树通过采伐来传播种子)  
      \item 可可树和当地的树有哪些不同特点和相同点?
    \end{itemize}  
    通过谷歌图片搜寻一下关于可可树的相关图片。\par
    下面的网址可能会提供有用的信息:\par
    \begin{itemize}
      \item \href{http://www.xocoatl.org/tree.htm}{xocoatl.com}Xocoatl网站提供有关可可树生命周期的信息和照片。
      \item \href{https://www.youtube.com/watch?v=LJ-1snuKJ7o}{www.youtube.com}- YouTube主持这个动画教育视频,解释了可可树如何成长。
    \end{itemize}  
    \par
    同学们可以画出一个可可树和当地一种树的不同的对比表格。也可以借机好好熟悉一下树的各个组成部分(根,叶子,树干,果实,花灯)\par
    他们应该可以指出两种树的不同和相同之处。比如:两种树的繁殖方式一样吗?他们都有果实吗?它们又是如何被分类的?它们属于一个群吗?每一种树又是如何繁殖的?而同学应该了解的是:这两种树是如何开花的?如何授粉繁殖的?\par
    高年级的同学可以自行画出一个有两栏的图标:相似点,不同点?在相应的区域写上不同的部分。\par
    低年级的同学可以找出当地一些树的照片和可可树的照片作对比。\par
    作为拓展部分,同学可以做一个实验来观察水是如何传输到植物/树中的。简单的准备一罐加了颜色的水,然后将一棵芹菜放进罐中。同学可以开始观察和记录随着时间发生在植物身上的事情————使用一个放大镜来进一步观察芹菜杆。(孩子们最终会看到,随着水的不同深入,芹菜杆的颜色会跟着加深,这就是水的足迹)。然后同学可以在自己的表格中加上根和枝干对植物的作用————比如通过根和枝干来传输水源和土壤中的营养的功能。


\section{个人目标}
    \begin{itemize}
      \item 适应
      \item 沟通 
      \item 探究
      \item 深思
    \end{itemize}  
