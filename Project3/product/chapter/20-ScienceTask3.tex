\chapter{科学任务-3}


\section{学习目标}
  \begin{itemize}
      \item 能够完成简单的分析。
      \item 能够简洁地,公平地,周全地分析每一次分析。 
      \item 能够对分析的结果做出预测。
      \item 能使用简单的科学设备。
      \item 能够通过观察和测量研究来对已有观点进行测试。
      \item 能够将观测到的现象和更加深刻的科学知识联系在一起。
      \item 能够通过观测到的现象看到本质的结果。
      \item 能够从简单的文本中收集到有用的信息。
      \item 明白收集科学证据的重要性。
      \item 了解到营养,生长,活动和繁殖的本质。
      \item 了解人类和其他动物牙齿的功能和防护。
   \end{itemize}  



\section{探究活动}
     经常吃巧克力的你有没有长过蛀牙?请同学将,并述说自己是如何发现的?利用这个机会回忆一下不同类型的牙齿及其功能。你的牙齿主要是的功能是啥?你的牙齿用来咬,撕,和咀嚼各种各样的食物。这些牙齿各自有不同的功能和不同的工作。孩子们知道这些牙齿的名字吗?\par
     切齿称为门牙(前牙),刺齿是犬齿(尖牙),咀嚼和挤压牙齿是臼齿(后牙)。 在完成这项任务的调查时,鼓励同学注意不同类型的牙齿。\par
     调查的一种方式就是测试巧克力和不同的高热量食物(见科学任务3)。哪种零食会留下最多的牙菌斑?\par
     将全班同学分组,每一个小组都有:\par
     \begin{itemize}
       \item 一块巧克力
       \item 一片新鲜的胡萝卜、苹果或者橘子
       \item 一些土豆薯片
       \item 水果切片
       \item 手镜
       \item 一次性牙刷(每个孩子一个)
       \item 牙膏
     \end{itemize}  
     首先让孩子们做一份预测。然后讨论他们如何设计这些休闲食品的公平测试。 他们怎么能测量牙齿上的牙菌斑? (使用平板电脑,拍照,绘制图表)。\par
     他们或许可以做一下下面的测试:\par
     \begin{itemize}
       \item 刷牙
       \item 吃一块巧克力
       \item 观察结果
       \item 吃一块切片
       \item 将结果照相
       \item 刷牙
     \end{itemize}
       为了公平起见,他们应该需要每一种水果都需要相同的步骤。

\section{记录活动}
   同学需要通过照相,拍视频或者注释表格,写描述的方式来记录结果。\par
   哪一种食物留下最多的牙斑菌?和之前预测的一样吗?问一下孩子们从这次实验中学到了什么?他们能应用自己所学的知识点来关爱自己的牙齿吗?

\section{个人目标}
   \begin{itemize}
     \item 沟通
     \item 探究
     \item 尊重
     \item 深思
   \end{itemize}  
