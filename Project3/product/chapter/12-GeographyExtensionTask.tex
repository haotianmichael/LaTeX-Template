\chapter{地理扩展任务}


\section{学习目标}
    \begin{itemize}
      \item 了解特定的地域是如何影响人类的行为的
      \item 了解特定地域的性质是如何影响人类的生活的
      \item 能够使用地理的专业术语
      \item 能够通过二级信息获取地理信息
      \item 能够清晰的表述关于一片自然环境的特征,以及如何对其进行改善。
      \item 能够谈论关于地理相关的话题,可以理解并回答一些关于地理的问题。
      \item 理解小地区是如何适应比较大的地理版图的
      \item 理解自然环境可以被破坏,当然也可以被修复。
    \end{itemize}  

    

\section{扩展任务}
     咖啡和巧克力有什么相同点?它们生长在相同的气候条件,也同样都是经济作物。\par
     "长在树上的钱"。像同学们介绍这句谚语。并适当的谈论一下关于经济作物的事情比如咖啡和可可豆。大面积的热带雨林被砍伐用例作为经济作物的产地。这种行为会不会对自然环境和当地的居民造成什么影响?\par
     下面的网站提供了一个很好的出发点:\par
     \begin{itemize}
       \item  \href{http://environment.nationalgeographic.com/environment/photos/rainforest-deforestation//%23/madagascar-slash-burn_278_600x450.jpg}{pictures.jpg}国家地理网站上有雨林砍伐森林的照片。
       \item  \href{http://www.rainforestsaver.org/what-slash-and-burn-farming}{rainforest.com}Rainforest Saver网站解释了什么是刀耕火种。
       \item  \href{http://www.edenproject.com/rainforest/}{edenproject.com}伊甸园项目网站解释了雨林的重要性。
       \item  \href{http://www.worldcocoafoundation.org}{worldcocoafoundatio.com}世界可可基金会网站提供有关可持续可可种植的信息和视频。
     \end{itemize}  
     
     同学可以小组内讨论并探究这些问题:\par
     \begin{itemize}
       \item '刀耕火种'——这些事情是在哪里,因为什么发生的?
       \item ’经济作物‘——经济作物的利弊?
       \item ‘森林砍伐’——这种行为会对地球的气候,植被和动物造成什么样的影响?
       \item ‘保守估计在2050年会有不到5\%的热带雨林留下来’——为什么我们需要拯救世界的热带雨林。
       \item ‘人类’——当地人还可以以什么为生?
     \end{itemize}  
     你可以根据每一个小组的能力不同,分配不同的工作给他们。\par
     同学可以通过话剧,诗歌或者歌曲的形式来展示自己的成果。最后全班一起讨论大肆破坏热带雨林种植经济作物的利弊是什么?\par
     下面的视频解释了可可树的有机种植如何拯救热带雨林:\par
     \begin{itemize}
       \item \href{http://video.nationalgeographic.com/video/player/news/environment-news/domrep-cacao-wcvin.html}{youtube.com}国家地理网站上有关于可可有机农业如何实际拯救多米尼加共和国热带雨林的视频。
     \end{itemize}  
     
     

     
\section{个人目标}
    \begin{itemize}
      \item 沟通
      \item 探究
      \item 道德
      \item 尊重
      \item 深思
    \end{itemize}  

