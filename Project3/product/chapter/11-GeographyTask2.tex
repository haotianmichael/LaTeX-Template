\chapter{地理任务-2}

\section{学习目标}
    \begin{itemize}
      \item 了解特定地区的性质是如何影响当地人们的生活的。
      \item 能够使用地理专业的术语
      \item 能够使用二级信息源获取地理信息
      \item 能够交谈关于地理的话题并且理解和回答一些关于地理相关的问题
      \item 了解小地域是如何适应比较大的地理版图的
    \end{itemize}  

\section{探究活动}
   你想要在一个种满可可树的农场里工作吗?你会吃很多的巧克力吗?让我们一起探究一下。\par
   从解释“经济作物”这个单词开始:就是这些产品被种不是用来当地消费的,而是赚钱用的。所以说那些中可可树的人是不会吃巧克力的!\par
   他们为什么不吃巧克力呢?邀请同学提出建议(巧克力会在热天气下融化,而且这些东西都很贵)。农场主将可可豆卖到其他的国家比如美国,瑞士,德国和比利时。这些国家(工业化程度更高和更加有钱的)负责加工和将这些可可豆量产化为巧克力。\par
   \begin{itemize}
     \item Cacao——树的名字
     \item Cocoa——种子里面的豆
     \item Cocoa exports——那些卖可可豆的国家(供应商) 
     \item Cocoa importers——那些买可可豆的国家(加工商)
   \end{itemize}   
   下面的一些网址会提供一些有用的链接和信息:\par
   \begin{itemize}
     \item \href{http://globaldimension.org.uk/news/item/14702}{globaldimension.com}Global Dimension提供探索巧克力生产全球方面的信息和视频。
     \item \href{http://www.sfu.ca/geog351fall03/groups-webpages/gp8/intro/intro.html}{geog351fall03.com}- 世界巧克力地图集探索巧克力生产和消费的地理位置。
     \item \href{http://ngm.nationalgeographic.com/ngm/0404/resources_geo2.html}{nationalgeographic.com}国家地理网站提供了有用的背景信息:巧克力之路(附带链接)。
   \end{itemize}
   高年级的孩子们可以探究下面的问题:\par
   \begin{itemize}
     \item 当地的人通过种植可可豆转到的利益往往很少
     \item 一些农场主甚至雇用童工,都是一些想通过打工来帮助自己的家庭度过困难的苦孩子。
   \end{itemize}  
   

\section{记录活动}
     请同学在普通的世界轮廓地图上定位一些巧克力的加工商所在国家。并在这些国家的边上标上名字(美国,英国,瑞士,德国,比利时等)。\par
     现在将两张地图放到一起————在任务1中做的可可豆的供应商所在的国家和这次任务中做的可可豆的加工商所在的国家。现在同学们可以很清楚的看见巧克力的供应商和加工商。将供应商和加工商之间的出口画上一条线,并用颜色分清供应商和加工商。\par
     高年级的同学能不能想出一个办法来解决低收入和童工的方法?如何才能使得他们的建议得到实施?老师也可以鼓励孩子们写信给一些主要的巧克力加工商问一问他们是如何处理这些事情的?或者参考国际化任务1。\par
     技术链接: 探究可可树是如何生长的,以及他们的加工流程和可可豆制作成为巧克力的流程。将这些步骤做成一个插图形式或者卡通动画。\par
     下面的网址可以提供一些有用的帮助:\par
     \begin{itemize}
        \item \href{http://www.dubble.co.uk/bean2bar}{dubble.com}Dubble网站为儿童提供信息和视频.
        \item \href{http://www.divinechocolate.com/uk/about-us/research-resources/divine-story/bean-to-bar}{divinechocolate.com}Divine Chocolate提供信息和照片,介绍从可可树到消费者的巧克力制作过程。
        \item \href{http://www.scharffenberger.com/our-story/artisan-process/}{scharffenberger.com}Scharffenberger网站解释了他们工厂生产巧克力的过程。
        \item \href{cocoaskiss.blogspot.com }{cocoaskiss.com} - Cocoaskiss网站为教师提供有关巧克力的所有信息和照片。
     \end{itemize}  
     


\section{个人目标}
    \begin{itemize}
      \item 沟通
      \item 探究
      \item 道德
      \item 深思  
    \end{itemize}  
   
