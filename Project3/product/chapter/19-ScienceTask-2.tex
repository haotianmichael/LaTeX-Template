\chapter{科学任务-2}
      
\section{学习目标}
  \begin{itemize}
    \item 能够从简单的文本中收集信息。
    \item 了解营养,生长,运动和繁殖的本质。
    \item 了解平很膳食对人体健康的重要性。
  \end{itemize}  

\section{探究活动}
   问一下同学为什么要吃巧克力?巧克力可以看成是饭点之间的零食吗?还是巧克力是主餐中的一部分?还是它是治疗餐中的一部分?同学平时常吃巧克力吗?每一天?每一周?还是旨在特殊场合吃?做一份班级调查来谈及孩子们吃巧克力的习惯,然后做一份为期一周的“吃巧克力日记”。\par
   你会发现巧克力更多的是被作为饭点之间的零食来吃的。\par
   科学任务的扩展部分需要高年级的孩子完成。比较一下巧克力和其他的高热量零食比如薯片、花生豆等中的营养成分。
  
\par 薯片: \par
 \begin{table}
     \begin{tabular}{l|l}
       \hline
       每100克的典型值
       热量 & 513千卡路里
       蛋白质 & 5.8克
       碳水化合物 &  51.5克
       蔗糖 &  0.5克
       油脂 & 30.1克
       纤维 & 6.5克
       钠 & 0.09克
       \hline
     \end{tabular}
   \end{table}

\par 花生豆 \par
   \begin{table}
     \begin{tabular}{l|l}
       \hline
       每100克的典型值
       热量 & 650克
       蛋白质 & 15.3克
       碳水化合物 & 14.4克
       蔗糖 & 5.0克
       油脂 & 58.9克
       纤维 & 6.1克
       钠 & 0克
       \hline
     \end{tabular}
   \end{table}
   
   通过比较食物包装袋上的标签,同学可以找到更多的高热量食物。通过这件事情,同学对零食的营养成分有什么样的新认识?是不是所有的零食都意味着不健康?
\section{记录活动}
    画出表格和饼图表格来帮助分析孩子们吃巧克力习惯和频率的结果。这份结果体现了什么?巧克力是作为饭点之间的零食来吃的吗?或者仅仅只是一种重要场合的装饰品?\par
    对这些零食包装袋上的标签排个序。以蔗糖含量为标准,以油脂为标准,两者都很高的。孩子们可以画出一副维恩图来展示结果。\par
    吃一袋薯片,吃一袋花生豆和吃一块巧克力,那个更加健康一些呢?问一下孩子们关于这个问题的看法。\par
    同学们可以制作一份表格,展示出不同的零食中各个营养成分的不同含量多少。\par
    这些东西应该在什么吃比较好?政府需要警告这些零食有健康隐患吗?比如使用交通灯标志————“红色”表示只能迟一点?(参见艺术和技术任务)。\par
    技术链接:选择一些食物,包括一些四季水果和蔬菜。同学探究这些食物在营养膳食中的角色。还可以使用这些食物来作出属于自己的营养餐,沙拉或者开胃菜(如奶酪/水果松饼)来代替巧克力。根据食谱的不同,同学可以继续探究这些不同的成分的味道。探究一下不同成分的不同搭配。在最后展示作品的时候,鼓励孩子们介绍一下自己使用的不同的成分以及使用这些成分的原因。需要单独有时间做一个“品尝时刻”,甚至可以请学校其他班级的学生和成员来品尝。然后做出评价。(不管是做还是品尝的任何时刻都应该清楚这这些食品中的成分)。



\section{个人目标}

  \begin{itemize}
    \item 沟通
    \item 合作
    \item 探究
    \item 道德 
    \item 深思
  \end{itemize}  
