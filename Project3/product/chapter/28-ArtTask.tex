\chapter{艺术任务}


\section{学习目标}


\begin{itemize}
  \item 了解一些艺术家,包括他们的祖国和居住的国家——使用形式,材料和流程以满足其目的
  \item 能够将艺术作为自我表达的一种方式
  \item 能够正确选择和此次任务相符的材料
  \item 能够根据他们的工作和原因来解释他们自己的工作
  \item 能够谈论艺术作品,给出他们意见的理由
\end{itemize}



\section{探究活动}
   通过录制巧克力棒的电视广告开始任务。\par
   请问同学,这些广告的受众是谁?小孩还是大人?男孩还是女孩?男人还是女人?\par
   广告试图说什么?它成功达到目的了吗?\par
   告诉孩子们这些广告商雇佣艺术家和设计家来创造电视和网络广告。像其他的艺术家一样,他们也会考虑很多事情————颜色,形状,形式和表达的技巧,通过这些方式来将这些东西最恰当和完美的展现给那些广告用户。\par
   收集一些巧克力的包装袋,谈论一下每一个包装袋的设计方式:\par
   \begin{itemize}
     \item 都是用了什么颜色,具体是如何搭配的?
     \item 这些颜色会冲突吗?看起来会比较多余吗?
     \item 字体上使用了哪种字体? 例如,加入脚本或间隔字母? 大小写?
     \item 过程中使用了什么形状,颜色,线条,字体和间距?
     \item 孩子们最喜欢哪一种巧克力包装袋?
   \end{itemize}   
   向孩子们指出,那些包装袋上的成分表和说明是法律要求的合理要求。比如:成分表,制作商的名称和地址,重量等等。\par
   

\section{记录活动}
   邀请同学为在技术任务-1中制作的巧克力设计一套包装袋。他们应该为这块新巧克力起一个好名字,并且思考颜色,形状,线条,形式和在设计上使用的插图等。\par
   ICT链接:孩子们可以使用设计软件为他们的巧克力棒创建并打印出包装纸。 他们应该首先测量巧克力的大小,并允许周围的空间能够折叠。 使用软件中提供的所有设计工具来选择字体,大小,颜色,图案等,以创建完成的设计。\par
   科学链接:应该在巧克力棒包装上加上“健康警告”吗? 想想科学任务3(巧克力的高能量含量)和任务4(牙齿卫生)。 因此,它可能会减少巧克力的销售,但要让消费者有知情权是一种责任。



\section{个人目标}


\begin{itemize}
  \item 沟通
  \item 探究
  \item 深思  
\end{itemize}
