\chapter{学习入口}

   \begin{note}
     在开始本单元的学习之前一定要检查好任何食物过敏的情况,尤其是一些类型的糖尿病和花生过敏。这种类型的病状后果通常非常严重,所以这一点怎么强调都不为过。有一些同学可能仅仅只是闻到了花生的味道就会有不舒服的表现。所以老师们如果发现有什么不对劲的地方,一定要及时就医。\par
     在教室设立一个感官实验室,就巧克力而言不同的感官要设立不同的试验区域。并用一些合适的图片来描述这些感官。并对每一个感官提出一两个有意思的问题,以便于同学在上课的时候可以以巧克力专家的身份或者品尝师的身份来分析和调查这些问题。比如:\par


\section{味道}
   \beign{itemize}
     \item 班上的同学是否可以通过品尝来鉴别黑巧克力、牛奶巧克力和白巧克力
     \item 同学们更喜欢吃甜味道的巧克力还是可口的奶酪
   \end{itemize}


\section{嗅觉}
    \begin{itemize}
      \item 如果巧克力闻起来是洋葱的味道我们是不是还喜欢它
      \item 如果我们闻不到巧克力的味道,那我们是不是还要去品尝
    \end{itemize}


\section{视觉}
     \begin{itemize}
        \item  为什么巧克力制造商们不制作绿色的和紫色的巧克力
        \item  如果巧克力看起来像卷心菜的样子,那阿门是不是还要品尝
     \end{itemize} 


\section{触觉}
       \begin{itemize}
         \item 巧克力最低的溶点是多少
         \item 在冰箱中的巧克力和常温下的巧克力味道是不是一样的
        \end{itemize} 

\section{听觉}
       \begin{itemize}
         \item 巧克力能发出声音吗
         \item 当我吃巧克力时,其他人可以听到我嘴里的砰砰声吗
       \end{itemize}  

       \par
       将每一块感官试验区域按小组分开。然后每个小组的同学开始自行寻找合适的测试方法来检测上面的问题,并得出问题的答案。下面的测试将以味道为例:\par

\section{测试}
      测试人员需要:\par
      \begin{itemize}
        \item 品尝师——选择自愿者,确保他们中不会出现对将要测试的食物过敏的情况
        \item 蒙住他们的眼睛
        \item 完全清洁的条件 - 这可能是一个科学测试,但它必须像任何其他餐一样对待
        \item 需要进行比较的食物。比如:
        \item 黑巧克力,切成块状
        \item 甜白巧克力,切成块状 
        \item 白巧克力,切成块状
        \item 薯片——风味或者盐腌的
        \item 花生——甜的和咸的
        \item 干果——切成块状
        \item 1.确保实验环境是完全卫生的
        \item 2.使用一个干净的厨房用刀,将食物切成相等的份
        \item 3.蒙上志愿者的眼睛,拿出两份样品,请他们品尝但每个人只能品尝一次.然后选出哪一个更好吃。
        \item 4.多测几次以确保实验的可靠性
        \item 5.以表格和图片的形式报告最后的结论,哪一种食物是最受迎的食物,选票的差距大吗
      \end{itemize} 
      

      
\section{注意}
      进行嗅觉的测试的时候,同学可以在正常的品尝之后,在捏住鼻子的基础上再进行一次品尝,便会发现很难会尝出什么味道。(回忆一下在感冒的时候,鼻子不通,是不是也是嘴里没有味道!)


\end{note}
   
