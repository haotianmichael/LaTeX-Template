\chapter{科学任务-4}


\section{学习目标}
    \begin{itemize}
      \item 能够完成简单的分析。
      \item 能够简洁地,公平地,周全地分析每一次分析。 
      \item 能够对分析的结果做出预测。
      \item 能使用简单的科学设备。
      \item 能够通过观察和测量研究来对已有观点进行测试。
      \item 能够将观测到的现象和更加深刻的科学知识联系在一起。
      \item 能够通过观测到的现象看到本质的结果。
      \item 能够从简单的文本中收集到有用的信息。
      \item 明白收集科学证据的重要性。
      \item 了解到营养,生长,活动和繁殖的本质。
      \item 了解人类和其他动物牙齿的功能和防护。
      \item 了解坚持锻炼对人体的重要性。
      \item 能够就简单事物的基本特性对他们作出比较。
      \item 理解不同的物质可能会适应不同的需求和目标
   \end{itemize}  


\section{探究活动}
    探究一下巧克力的包装袋。主要是什么成分组成的?为什么会使用这种物质?这种物质会有啥特性?(防止巧克力融化,确保巧克力新鲜,保持其他气味和口味)。 \par
    邀请孩子们计划一次分析,来测试一下使用不同的物质制作巧克力的包装袋的效果。比如:\par
    同学可以将巧克力放到下面的包装中:  \par
    \begin{itemize}
      \item 防油或蜡纸
      \item 厨房铝箔
      \item 纸巾
      \item 保鲜膜
      \item 标准的巧克力包装纸
    \end{itemize}  
    步骤:\par
    \begin{itemize}
      \item 1.将它们密封在聚乙烯三明治袋中,其中含有强烈气味的食物,例如: 洋葱片或压碎的大蒜
      \item 2.将一块巧克力包裹在标准巧克力包装中 - 这是测试中的“控制”
      \item 3.放置几天
      \item 4.预测会发生什么
      \item 5.打开巧克力,品尝它 
      \item 6.哪个包装好了? 你能解释一下原因吗?
    \end{itemize}  
    和高年级的学生一起完成本任务的扩展部分————可以从上面的包装带中找出一种对巧克力具有绝缘功能的包装材料(比如放置巧克力融化)。同学挑战多个不同的测试。


\section{记录活动}
   孩子们可以通过一份写注释的表格或者视频的形式完成自己的分析报告。\par
   从这些测试中,他们能得出什么结论?这个结果公平吗?比如:他们是否在一个包装袋中使用了相同量的洋葱和大蒜?结果有没有效等。


\section{个人目标}
 \begin{itemize}
   \item 沟通
   \item 探究
   \item 坚韧
   \item 深思
 \end{itemize} 
