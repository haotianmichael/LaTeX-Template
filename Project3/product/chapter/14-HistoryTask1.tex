\chapter{历史单元任务-1}

\section{学习目标}
     \begin{itemize}
       \item 通过研究了解到一些关于古时代的主要大事件、纪时和历史人物。
       \item 了解早某个特定时期人们的生活。 
       \item 能够给出一些特定时间的具体原因和转折点及其原因。
       \item 能够从简单的来源收集信息。
       \item 能够使用自己所学的知识和理解对过去做出合理的解释。
     \end{itemize}  


\section{探究活动}
   人类喝巧克力的历史已经有上千年的了————不是那种在超级市场见到的“热巧克力”,而是一些叫做“xocolatl”或者“chocolatl”的东西。\par
   问一下同学,以小组为单位探究巧克力的历史,下面是主要需要关注的问题:\par
   \begin{itemize}
      \item 什么是“chocolatl”?谁制作了它?是如何做的?
      \item 谁第一次喝,当时的情况是怎么样的?
      \item 它和我们现在喝的巧克力有什么不一样?
      \item 哥伦布第一次将可可豆带回到西班牙的时候,可可豆为什么被忽视了?
      \item 哪一位西班牙的探险家第一次大面积种植和栽培可可豆呢?  
      \item 之后又发生了什么?
   \end{itemize}  

   下面的书籍和网址提供了相应的帮助:\par
   \begin{itemize}
     \item The Story of Chocolate, by Caryn Jenner, Dorling Kindersley, 2005
     \item The Story of Chocolate, by Katie Daynes, Usborne Publishing, 2006
     \item \href{http://mexicolore.co.uk/maya/chocolate/}{mexicolore.com}- 这个教育网站专门关注巧克力(可可)在阿兹特克人和玛雅人中的作用,文章适用为老师和孩子们。
     \item \href{http://www.exploratorium.edu/exploring/exploring_chocolate/choc_2.html}{exploratorium.com}Exploratorium网站探索巧克力的历史。
     \item \href{http://www.inventors.about.com/od/foodrelatedinventions/a/chocolate.htm}{inventors.com}- About.com网站有可可豆的历史时间表。 
     \item \href{http://www.thestoryofchocolate.com}{thestoryofchocolate.com}“巧克力的故事”中有一个“谁依赖它?”部分,其中包含有关巧克力对过去人们的影响的信息
     \item \href{http://www.sfu.ca/geog351fall03/groups-webpages/gp8/history/history.html}{sfu.ca.com}SFU网站包含有关巧克力历史的有用信息。
   \end{itemize}  
   

\section{记录活动}
     以小组为单位,请孩子们利用自己探究的成果写一系列短话剧和小插曲,通过这种形式来表现巧克力的发展历史。比如:   \par

     \begin{itemize}
       \item  从阿兹特克国王蒙特祖玛开始,情景就是为自己的客人在提供放在金杯中的可可豆。
       \item  哥伦布向西班牙的国王和王后展示了珠宝、金银和大量的可可豆,但是国王和王后对可可豆视而不见。
       \item  这时候科尔特斯上场,他不喜欢喝可可豆煮的饮料,但是他发现可可豆竟然被用来作为货币使用!
       \item  科尔斯特回到西班牙之后,带回了可可豆。这时候,西班牙人在饮料中添加了白糖而不是香料,一夜之间所有人都爱上了巧克力!
       \item  但是不是所有的人都负担得起和巧克力的费用。只有欧洲的王室贵族,富人和国王才能享受这种殊荣。
     \end{itemize}  

     等等。\par
     最后,同学们应该画出一幅有插图的时间轴,这幅画上带有每一个当事人的名字和事件发生的时间。这些事件需要从蒙特妈祖引出可可豆开始记录一直到今天我们用钱买巧克力。\par
     艺术链接:研究阿兹特克艺术。 用一个饮用杯形状的粘土做一个简单的捏锅或线圈罐。 烘烤,装饰和上釉。 使用黄色作为基础,并用黑色,棕色和白色几何图案装饰。 然后在烘烤。\par
     下面的网址提供了一个比较好的出发点。\par
     \begin{itemize}
       \item \href{kinderart.com/sculpture/clay.shtml }{kinderart.com}Kinderart网站提供有关制作陶器的教师信息。
     \end{itemize}  




\section{个人目标}
   \begin{itemize}
     \item 适应
     \item 沟通
     \item 探究
     \item 深思
   \end{itemize}   
