\chapter{科学扩展任务}

\section{学习目标}
     \begin{itemize}
      \item 能够完成简单的分析。
      \item 能够简洁地,公平地,周全地分析每一次分析。 
      \item 能够对分析的结果做出预测。
      \item 能使用简单的科学设备。
      \item 能够通过观察和测量研究来对已有观点进行测试。
      \item 能够将观测到的现象和更加深刻的科学知识联系在一起。
      \item 能够通过观测到的现象看到本质的结果。
      \item 能够从简单的文本中收集到有用的信息。
      \item 明白收集科学证据的重要性。
      \item 了解到营养,生长,活动和繁殖的本质。
      \item 了解人类和其他动物牙齿的功能和防护。
      \item 了解坚持锻炼对人体的重要性。
      \item 能够就简单事物的基本特性对他们作出比较。
   \end{itemize}  

\section{扩展任务}
     在每一位同学的手中放一块巧克力。观察所发生的事情。巧克力马上开始融化了吗?为什么它会融化?(手掌中的温度比巧克力的温度要高)它会完全融化吗?如果不是,为什么?\par
     请同学思考是否还有什么物质可以在手掌心中融化(黄油、冰激凌等)。\par
     接着同学可以开始自己的分析实验,来探究哪一种巧克力的溶点最低。黑巧克力、白巧克力还是牛奶巧克力?\par
     每一个小组将会需要:\par
     \begin{itemize}
        \item 一块黑巧克力、白巧克力和牛奶巧克力。
        \item 计时器
        \item 烹饪温度计(可选) 
        \item 一碗热水(注意:不要比自来水太热)
        \item 三个铝箔蛋糕盒或'船'            
     \end{itemize}  
     \par步骤:\par
     \begin{itemize}
       \item 每次试验在蛋糕盒上放一块巧克力
       \item 将蛋糕盒放到水面上
       \item 观察巧克力融化的过程和状态变化
       \item 冷却水温
       \item 观察巧克力又变成固态的过程
     \end{itemize}
     
     在开始这次分析之前,老师应该首先请同学做一次预测,然后决定他们应该使用什么测量方式。比如:巧克力开始融化的时间,巧克力完全融化的时间,他变成固态的时间。\par
     高年级的学生还可以将这些变化的温度画成一张折线图。\par
     为了方便在学习出口的时候和父母交流,可以考虑拍视频。\par
     孩子们应该在条形图中记录融化的顺序和时间。 他们应该能够使用“固体”,“固化”和“液体”,“液化”等术语,并将这些术语应用于其他食物。\par
     这项活动将向他们展示固体如何变成液体并再次变成固态。 同学能否解释此现象? 这适用于所有固体吗? 他们能想到以这种方式变化的其他食物吗? (黄油)\par
     然后,孩子们可以继续研究不同材料通过加热和冷却的变化状态,以及发生这种情况的温度(以摄氏度,摄氏度为单位)。在这项任务中,孩子们一直在使用“固体”和“液体”这两个术语,但是他们能说出第三种物质状态吗? (气体)孩子们可以想到任何固体,液体和气体材料的例子吗? 在白板上绘制一个三柱图表:固体,液体和气体,并要求孩子们用示例填写图表。 查看每列中的材料并讨论它们的共同点。 例如,我们可以感觉到固体并且它们具有形状。 液体流动,我们可以倒它们。 气体通常是看不见的,我们通常不会感觉到它们,但我们有时会闻到它们。\par
     同学或许已经熟悉了水可以有三种存在的状态,加热的时候可以变成气态。如果同学还没有这方面的知识,老师可能需要向同学们介绍水循环的过程,以及蒸发和冷凝所起的作用,将蒸发速率与温度联系起来(有关这方面的更多信息,以及固体,液体和 气体请参见Milepost 2 unit Material World)。\par
     如果我们继续加热巧克力,它会变成气态吗?听听大家的意见,然后全班一起探究。\par
     安全注意事项:教师应负责加热巧克力,在此活动期间应密切监督儿童。
     
     


\section{个人目标}
    \begin{itemize}
      \item 沟通
      \item 探究
      \item 深思
    \end{itemize}  
